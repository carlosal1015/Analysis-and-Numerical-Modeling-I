%! Aldo Luna, Alexandra Mazzetti, Carlos Aznarán, Edward Canales
%! Universidad Nacional de Ingeniería
%! Facultad de Ciencias
%! Lima, Perú
%! Uso:
%! $ sudo pacman -Syu texlive-most zathura # dependencias, visor
%! $ arara solution_tercera
%! $ zathura solution_tercera.pdf
%! Ver https://wiki.archlinux.org/title/TeX_Live
% arara: clean: {
% arara: --> extensions:
% arara: --> ['aux','log','nav',
% arara: --> 'out','snm','toc','pytxcode','pdf']
% arara: --> }
% arara: lualatex: {
% arara: --> shell: yes,
% arara: --> draft: yes,
% arara: --> interaction: batchmode
% arara: --> }
% arara: biber
%! arara: pythontex
% arara: lualatex: {
% arara: --> shell: yes,
% arara: --> draft: yes,
% arara: --> interaction: batchmode
% arara: --> }
% arara: lualatex: {
% arara: --> shell: yes,
% arara: --> synctex: yes,
% arara: --> interaction: batchmode
% arara: --> }
% arara: clean: {
% arara: --> extensions:
% arara: --> ['aux','log','nav',
% arara: --> 'out','snm','toc','pytxcode']
% arara: --> }
\PassOptionsToPackage{svgnames}{xcolor}
\documentclass[
	spanish,
	8pt,
	utf8,
	xcolor=table,
	handout,
	aspectratio=169,
	professionalfonts,
	% notheorems,
	mathserif,
	leqno,
	% t
]{beamer}
\setbeamersize{text margin left=5pt,text margin right=5pt}
\usepackage[spanish,es-sloppy]{babel}
\spanishdatedel\decimalpoint
\usepackage{mathtools}
\usepackage{nicematrix}
\usepackage{systeme}
\usepackage{minted}
\usepackage{enumerate}
\usepackage{multicol}
% \usepackage{pythontex}
\usepackage[
	citestyle=numeric,
	style=apa,
	backend=biber,
	defernumbers=true,
	sorting=ynt,
	maxcitenames=4
]{biblatex}
\addbibresource{references.bib}

\newcolumntype{x}[1]{>{\centering\arraybackslash\hspace{0pt}}p{#1}}

\newcounter{savedenum}
\newcommand*{\saveenum}{\setcounter{savedenum}{\theenumi}}
\newcommand*{\resume}{\setcounter{enumi}{\thesavedenum}}

\setbeamertemplate{navigation symbols}{}
\setbeamertemplate{footline}{}
\setbeamertemplate{headline}{}

% https://tex.stackexchange.com/questions/68080/beamer-bibliography-icon
\setbeamertemplate{bibliography item}{%
	\ifboolexpr{ test {\ifentrytype{book}} or test {\ifentrytype{mvbook}}
		or test {\ifentrytype{collection}} or test {\ifentrytype{mvcollection}}
		or test {\ifentrytype{reference}} or test {\ifentrytype{mvreference}} }
	{\setbeamertemplate{bibliography item}[book]}
	{\ifentrytype{online}
		{\setbeamertemplate{bibliography item}[online]}
		{\setbeamertemplate{bibliography item}[article]}}%
	\usebeamertemplate{bibliography item}}

\defbibenvironment{bibliography}
{\list{}
	{\settowidth{\labelwidth}{\usebeamertemplate{bibliography item}}%
		\setlength{\leftmargin}{\labelwidth}%
		\setlength{\labelsep}{\biblabelsep}%
		\addtolength{\leftmargin}{\labelsep}%
		\setlength{\itemsep}{\bibitemsep}%
		\setlength{\parsep}{\bibparsep}}}
{\endlist}
{\item}

\makeatletter
\newenvironment<>{proofs}[1][\proofname]{%
    \par
    \def\insertproofname{#1\@addpunct{.}}%
    \usebeamertemplate{proof begin}#2}
  {\usebeamertemplate{proof end}}
\makeatother


\title{
	\huge\sffamily
	Tercera Práctica Dirigida\quad Grupo N$^{\circ}$~3
}

\subtitle{
	\large\scshape
	Análisis y Modelamiento Numérico I\quad CM4F1 B\\[.5\baselineskip]
		\normalsize\normalfont
		Profesor Jonathan Munguia La Cotera
}

\author{
	Aldo Luna Bueno\quad\and\quad
	Alexandra Gutierrez Mazzetti\quad\and\quad
	Carlos Aznarán Laos\quad\and\quad
	Edward Canales Yarin
}

\institute{\large
	Facultad de Ciencias \and
	Universidad Nacional de Ingeniería
}
\date{26 de octubre del 2022}

\begin{document}

\frame{\titlepage}

\begin{frame}
	\frametitle{Factorizaciones $LU$}
	Suponga que $A$ se puede factorizar en el producto de una matriz
	triangular inferior $L$ y una matriz triangular superior $U$.
	Entonces para resolver el sistema de ecuaciones $A x=b$ basta con
	resolver este problema en dos etapas:
	\begin{equation*}
		Lz=b\text{ resolver para }z,\qquad
		Ux=z\text{ resolver para }x.
	\end{equation*}
	Comenzamos con una matriz $A$ de orden $n\times n$ y buscamos
	matrices
	\begin{equation*}
		L=
		\begin{bNiceMatrix}
			\ell_{11}  & 0          & 0          & \cdots & 0          \\
			\ell_{21}  & \ell_{22}  & 0          & \cdots & 0          \\
			\ell_{31}  & \ell_{32}  & \ell_{33}  & \cdots & 0          \\
			\vdots     & \vdots     & \vdots     & \ddots & \vdots     \\
			\ell_{n 1} & \ell_{n 2} & \ell_{n 3} & \cdots & \ell_{n n}
		\end{bNiceMatrix}\text{ y }
		U=
		\begin{bNiceMatrix}
			u_{11} & u_{12} & u_{13} & \cdots & u_{1 n} \\
			0      & u_{22} & u_{23} & \cdots & u_{2 n} \\
			0      & 0      & u_{33} & \cdots & u_{3 n} \\
			\vdots & \vdots & \vdots & \ddots & \vdots  \\
			0      & 0      & 0      & \cdots & u_{n n}
		\end{bNiceMatrix}
	\end{equation*}
	tal que $A=LU$.
	Cuando esto es posible, decimos que $A$ tiene descomposición $LU$.
	Resulta que $L$ y $U$ no son únicamente determinados.
	De hecho, para cada $i$, podemos asignar un valor no nulo o bien
	para $\ell_{ii}$ o bien para $u_{ii}$.
	Por ejemplo, una elección simple es asignar $\ell_{ii}=1$ para
	$i\in\left\{1,\ldots,n\right\}$, haciendo así $L$ una matriz
	triangular inferior unitaria.

	\begin{theorem}[Existencia de la descomposición $LU$]
		Sea $A\in\mathbb{R}^{n\times n}$.
		Si todas las matrices esquina de $A$ son no singulares, entonces
		$A$ tiene una descomposición $LU$.
	\end{theorem}
\end{frame}

\begin{frame}
	\begin{proofs}
		Recordemos que la $k$-ésima matriz esquina de $A$ es la matriz
		\begin{equation*}
			A_{k}=
			\begin{bNiceMatrix}
				a_{11}  & a_{12}  & \cdots & a_{1 k} \\
				a_{21}  & a_{22}  & \cdots & a_{2 k} \\
				\vdots  & \vdots  & \ddots & \vdots  \\
				a_{k 1} & a_{k 2} & \cdots & a_{k k}
			\end{bNiceMatrix}
		\end{equation*}
		Sean $A_{k}$, $L_{k}$, y $U_{k}$ la $k$-ésima matriz esquina de
		las matrices $A, L$, y $U$, respectivamente.
		Nuestra hipótesis es que $A_{1},A_{2},\ldots,A_{n}$ son no
		singulares.

		A los efectos de una demostración inductiva, suponga que
		$L_{k-1}$ y $U_{k-1}$ se han obtenido.
		Si $i,j\in\left\{1,\ldots,k-1\right\}$, entonces $s$ también
		está.
		Por eso,
		\begin{equation*}
			A_{k-1}=
			L_{k-1}U_{k-1}.
		\end{equation*}
		Ya que $A_{k-1}$ es no singular por hipótesis, $L_{k-1}$ y
		$U_{k-1}$ son también no singulares.
		Dado que $L_{k-1}$ es no singular, podemos resolver el sistema
		\begin{equation*}
			\sum_{s=1}^{k-1}
			\ell_{is}u_{sk}=
			a_{\imath k},\quad i\in\left\{1,\ldots,k-1\right\}
		\end{equation*}
		para $u_{sk}$ con $s\in\left\{1,\ldots,k-1\right\}$.
	\end{proofs}
\end{frame}

\begin{frame}
	\begin{proof}[\proofname\ (Cont.)]
		Estos elementos caen en la $k$-ésima columna de $U$.
		Dado que $U_{k-1}$ es no singular, podemos resolver el sistema
		\begin{equation*}
			\sum_{s=1}^{k-1}
			\ell_{ks}u_{si}=
			a_{kj},\quad j\in\left\{1,\ldots,k-1\right\}
		\end{equation*}
		para $\ell_{ks}$ con $s\in\left\{1,\ldots,k-1\right\}$.
		Estos elementos caen en la $k$-ésima fila de $L$.
		Del requerimiento
		\begin{equation*}
			a_{kk}=
			\sum_{s=1}^{k}\ell_{ks}u_{s k}=
			\sum_{s=1}^{k-1}
			\ell_{k s}u_{s k}+\ell_{kk}u_{kk}
		\end{equation*}
		podemos obtener $u_{kk}$ dado que $\ell_{kk}$ ha sido considerado
		como unidad.
		Así, se han definido todos los nuevos elementos necesarios para
		formar $L_{k}$ y $U_{k}$.
		La inducción se completa observando que $\ell_{11}u_{11}=a_{11}$
		y, por tanto, $\ell_{11}=1$ y $u_{11}=a_{11}$.
	\end{proof}
\end{frame}

\begin{frame}

	\begin{enumerate}\setcounter{enumi}{1}
		\item

		      Considere la matriz
		      \begin{math}
			      \begin{bNiceMatrix}
				      2 & 2 & 1 \\
				      1 & 1 & 1 \\
				      3 & 2 & 1
			      \end{bNiceMatrix}
		      \end{math}.

		      \begin{enumerate}[a)]

			      \item

			            Demuestre que $A$ no se puede factorizar en el
			            producto de una matriz triangular inferior
			            unitaria y una matriz triangular superior.

			      \item

			            Intercambie las filas de $A$ para que esto se pueda
			            hacer.
		      \end{enumerate}

	\end{enumerate}

	\begin{solution}
		\begin{enumerate}[a)]

			\item

			      Decimos que la matriz $A$
			      \alert{no se puede factorizar por el método $LU$}
			      porque $\exists k_{0}=2\in\left\{1,2,3\right\}$ tal que
			      $A_{k_{0}}$ es singular, ya que
			      \begin{math}
				      \det\left(A_{2}\right)=
				      \begin{vNiceMatrix}
					      2 & 2 \\
					      1 & 1 \\
				      \end{vNiceMatrix}=
				      0
			      \end{math}.

			\item

			      Intercambiando la \alert{fila 2} con la \alert{fila 3}
			      obtenemos
			      \begin{math}
				      \widetilde{A}=
				      \begin{bNiceMatrix}
					      2 & 2 & 1 \\
					      3 & 2 & 1 \\
					      1 & 1 & 1
				      \end{bNiceMatrix}
			      \end{math}.
			      Además, $\widetilde{A}$ tiene una descomposición $LU$
			      porque
			      \begin{math}
				      \forall k\in\left\{1,\ldots,3\right\}:
				      \widetilde{A}_{k}
			      \end{math}
			      es no singular.
			      En efecto,
			      \begin{math}
				      \det\left(\widetilde{A_{1}}\right)=
				      \begin{vNiceMatrix}
					      2
				      \end{vNiceMatrix}=2\neq0
			      \end{math},
			      \begin{math}
				      \det\left(\widetilde{A_{2}}\right)=
				      \begin{vNiceMatrix}
					      2 & 2 \\
					      3 & 2
				      \end{vNiceMatrix}=-2\neq0
			      \end{math}
			      y
			      \begin{math}
				      \det\left(\widetilde{A_{3}}\right)=
				      \begin{vNiceMatrix}
					      2 & 2 & 1 \\
					      3 & 2 & 1 \\
					      1 & 1 & 1
				      \end{vNiceMatrix}=-1\neq0
			      \end{math}.

			      Por la existencia de la descomposición $LU$,
			      $\exists L,U\in\mathbb{R}^{3\times 3}$ tal que
			      \begin{equation*}
				      \begin{bNiceMatrix}
					      2 & 2 & 1 \\
					      3 & 2 & 1 \\
					      1 & 1 & 1
				      \end{bNiceMatrix}=
				      \begin{bNiceMatrix}
					      \ell_{11} & 0         & 0         \\
					      \ell_{21} & \ell_{22} & 0         \\
					      \ell_{31} & \ell_{32} & \ell_{33}
				      \end{bNiceMatrix}
				      \begin{bNiceMatrix}
					      u_{11} & u_{12} & u_{13} \\
					      0      & u_{22} & u_{23} \\
					      0      & 0      & u_{33}
				      \end{bNiceMatrix}
			      \end{equation*}
			      con $l_{ii}=1$.
		\end{enumerate}
	\end{solution}
\end{frame}

\begin{frame}
	\begin{solution}
		Operando por matrices elementales de modo que triangulicemos a
		partir de la matriz $\widetilde{A}$.
		\begin{align*}
			\widetilde{A}
			 & =
			\begin{bNiceMatrix}
				2 & 2 & 1 \\
				3 & 2 & 1 \\
				1 & 1 & 1
			\end{bNiceMatrix}    \\
			f_{21}\left(-\frac{3}{2}\right)\widetilde{A}
			 & =
			\begin{bNiceMatrix}
				2 & 2  & 1            \\
				0 & -1 & -\frac{1}{2} \\
				1 & 1  & 1
			\end{bNiceMatrix} \\
			f_{31}\left(-\frac{1}{2}\right)
			f_{21}\left(-\frac{3}{2}\right)\widetilde{A}
			 & =
			\begin{bNiceMatrix}
				2         & 2         & 1            \\
				\alert{0} & -1        & -\frac{1}{2} \\
				\alert{0} & \alert{0} & \frac{1}{2}
			\end{bNiceMatrix}=U.
		\end{align*}
		Por otro lado,
		\begin{math}
			L=
			f_{31}\left(\frac{1}{2}\right)
			f_{21}\left(\frac{3}{2}\right)=
			\begin{bNiceMatrix}
				1           & 0 & 0 \\
				0           & 1 & 0 \\
				\frac{1}{2} & 0 & 1
			\end{bNiceMatrix}
			\begin{bNiceMatrix}
				1           & 0 & 0 \\
				\frac{3}{2} & 1 & 0 \\
				0           & 0 & 1
			\end{bNiceMatrix}=
			\begin{bNiceMatrix}
				1           & 0 & 0 \\
				\frac{3}{2} & 1 & 0 \\
				\frac{1}{2} & 0 & 1
			\end{bNiceMatrix}.
		\end{math}
	\end{solution}
\end{frame}

\begin{frame}[fragile]
	\begin{minipage}{0.45\textwidth}
		\begin{listing}[H]
			\inputminted[
				fontsize=\scriptsize,
				breaklines,
				firstline=5,
				lastline=20
			]{python}{lualdo.py}
			\inputminted[
				fontsize=\scriptsize,
			]{text}{output.txt}
		\end{listing}
	\end{minipage}
	\begin{minipage}{0.45\textwidth}
		\begin{listing}[H]
			\inputminted[
				fontsize=\scriptsize,
				breaklines,
				firstline=23,
				lastline=51
			]{python}{lualdo.py}
		\end{listing}
	\end{minipage}
\end{frame}

\begin{frame}
	\begin{enumerate}\setcounter{enumi}{6}
		\item

		      Una compañía minera trabaja en 3 minas, cada una de las
		      cuales produce minerles de tres clases.
		      La primera mina puede producir 4 toneladas del mineral A, 3
		      toneladas del mineral B, y 5 toneladas del mineral C; la
		      segunda mina puede producir 1 tonelada de cada uno de los
		      minerales y la tercera mina, 2 toneladas del A, 4 toneladas
		      del B y 3 toneladas del C, por cada hora de funcionamiento.
		      Se desea satisfacer los tres pedidos siguientes
		      \begin{table}[ht!]
			      \centering
			      \tiny
			      \begin{tabular}{|c|c|c|c|}
				      \hline
				      Pedidos & Mineral A & Mineral B & Mineral C \\
				      \hline
				      $P_{1}$ & $19$      & $25$      & $25$      \\
				      \hline
				      $P_{2}$ & $13$      & $16$      & $16$      \\
				      \hline
				      $P_{3}$ & $8$       & $12$      & $10$      \\
				      \hline
			      \end{tabular}
		      \end{table}

		      Determine
		      \begin{multicols}{2}

			      \begin{enumerate}[a)]
				      \item

				            Modele el sistema a resolver.

				      \item

				            Resolver usando los programas $LDL^{T}$ y
				            Cholesky.
			      \end{enumerate}
		      \end{multicols}

	\end{enumerate}

	\begin{solution}
		\begin{minipage}{0.65\textwidth}
			\begin{enumerate}[a)]

				\item

				      Supongamos que $x_{i}$ es el número de horas de
				      funcionamiento de la $i$-ésima mina.

				      En total, se requiere
				      \alert{40 toneladas del mineral A},
				      \alert{53 toneladas del mineral B} y
				      \alert{51 toneladas del mineral C}.

				      Escribiendo de manera apropiada, obtenemos el
				      sistema lineal $AX=b$, donde
				      \begin{math}
					      A=
					      \begin{bNiceMatrix}
						      4 & 1 & 2 \\
						      3 & 1 & 4 \\
						      5 & 1 & 3
					      \end{bNiceMatrix}
				      \end{math},
				      \begin{math}
					      b=
					      \begin{bNiceMatrix}
						      40 \\
						      53 \\
						      51
					      \end{bNiceMatrix}
				      \end{math}
				      y
				      \begin{math}
					      X=
					      \begin{bNiceMatrix}
						      x_{1} \\
						      x_{2} \\
						      x_{3}
					      \end{bNiceMatrix}
				      \end{math}.

			\end{enumerate}
		\end{minipage}
		\begin{minipage}{0.25\textwidth}
			\begin{figure}[ht!]
				\centering
				\includegraphics[width=.25\paperwidth]{minerals}
				\caption{Minerales A, B y C, respectivamente.}
			\end{figure}
		\end{minipage}
	\end{solution}
\end{frame}

\begin{frame}
	\begin{solution}
		La matriz $A$ no tienen todo sus autovalores positivo, o sea, $A$
		no es definida positiva.
		Por lo que multiplicamos por la izquierda por $A^{T}$, resultando
		\begin{align*}
			AX
			 & =
			b                  \\
			A^{T}AX
			 & =
			A^{T}b             \\
			{\begin{bNiceMatrix}
				 4 & 1 & 2 \\
				 3 & 1 & 4 \\
				 5 & 1 & 3
			 \end{bNiceMatrix}}^{T}
			\begin{bNiceMatrix}
				4 & 1 & 2 \\
				3 & 1 & 4 \\
				5 & 1 & 3
			\end{bNiceMatrix}
			\begin{bNiceMatrix}
				x_{1} \\
				x_{2} \\
				x_{3}
			\end{bNiceMatrix}
			 & =
			{\begin{bNiceMatrix}
				 4 & 1 & 2 \\
				 3 & 1 & 4 \\
				 5 & 1 & 3
			 \end{bNiceMatrix}}^{T}
			\begin{bNiceMatrix}
				40 \\
				53 \\
				51
			\end{bNiceMatrix} \\
			\begin{bNiceMatrix}
				50 & 12 & 35 \\
				12 & 3  & 9  \\
				35 & 9  & 29
			\end{bNiceMatrix}
			\begin{bNiceMatrix}
				x_{1} \\
				x_{2} \\
				x_{3}
			\end{bNiceMatrix}
			 & =
			\begin{bNiceMatrix}
				574 \\
				144 \\
				445
			\end{bNiceMatrix} \\
			\widetilde{A}X
			 & =
			\widetilde{b}
		\end{align*}
		Ahora, $\widetilde{A}$ es una matriz simétrica porque
		$\widetilde{A}^{T}=\widetilde{A}$ y definida positiva porque
		cada autovalor de $\widetilde{A}^{T}$: $\lambda_{1}\approx 79.001$,
		$\lambda_{2}\approx 2.9600$, $\lambda_{3}\approx 0.038$ es
		positivo.
	\end{solution}
\end{frame}

\begin{frame}
	\frametitle{Factorización $LDL^{T}$}
	\begin{solution}
		La factorización $LDL^{T}$ tiene números 1 en la diagonal de la
		matriz triangular inferior $L$ por lo que necesitamos tener
		\begin{align*}
			\widetilde{A}
			 & =
			LDL^{T}                    \\
			\begin{bNiceMatrix}
				a_{11} & a_{21} & a_{31} \\
				a_{21} & a_{22} & a_{32} \\
				a_{31} & a_{32} & a_{33}
			\end{bNiceMatrix}
			 & =
			\begin{bNiceMatrix}
				1         & 0         & 0 \\
				\ell_{21} & 1         & 0 \\
				\ell_{31} & \ell_{32} & 1
			\end{bNiceMatrix}
			\begin{bNiceMatrix}
				d_{1} & 0     & 0     \\
				0     & d_{2} & 0     \\
				0     & 0     & d_{3}
			\end{bNiceMatrix}
			{\begin{bNiceMatrix}
				 1         & 0         & 0 \\
				 \ell_{21} & 1         & 0 \\
				 \ell_{31} & \ell_{32} & 1
			 \end{bNiceMatrix}}^{T} \\
			\begin{bNiceMatrix}
				50 & 12 & 35 \\
				12 & 3  & 9  \\
				35 & 9  & 29
			\end{bNiceMatrix}
			 & =
			\begin{bNiceMatrix}
				d_{1}          & d_{1}\ell_{21}                         & d_{1}\ell_{31}                              & \\
				d_{1}\ell_{21} & d_{2}+d_{1}\ell_{21}^{2}               & d_{2}\ell_{32}+d_{1}\ell_{21}\ell_{31}      & \\
				d_{1}\ell_{31} & d_{1}\ell_{21}\ell_{31}+d_{2}\ell_{32} & d_{1}\ell_{31}^{2}+d_{2}\ell_{32}^{2}+d_{3} &
			\end{bNiceMatrix}
		\end{align*}
		Por lo tanto,
		\begin{align*}
			a_{11}:50=d_{1}                                                       & \implies\boxed{d_{1}=50}            &
			a_{21}:12=\alert{d_{1}}\ell_{21}                                      & \implies\ell_{21}=\frac{12}{50}     & \\
			a_{31}:35=\alert{d_{1}}\ell_{31}                                      & \implies\ell_{31}=\frac{35}{50}     &
			a_{22}:3=d_{2}+\alert{d_{1}}\ell_{21}^{2}                             & \implies\boxed{d_{2}=\frac{6}{50}}  & \\
			a_{32}:9=\alert{d_{1}}\ell_{21}\ell_{31}+\alert{d_{2}}\ell_{32}       & \implies\ell_{32}=5                 &
			a_{33}:29=\alert{d_{1}}\ell_{31}^{2}+\alert{d_{2}}\ell_{32}^{2}+d_{3} & \implies\boxed{d_{3}=\frac{75}{50}} &
		\end{align*}
	\end{solution}
\end{frame}

\begin{frame}
	\begin{solution}
		Así, obtenemos
		\begin{align*}
			\widetilde{A}
			 & =
			LDL^{T}                    \\
			\begin{bNiceMatrix}
				50 & 12 & 35 \\
				12 & 3  & 9  \\
				35 & 9  & 29
			\end{bNiceMatrix}
			 & =
			\begin{bNiceMatrix}
				1         & 0         & 0 \\
				\ell_{21} & 1         & 0 \\
				\ell_{31} & \ell_{32} & 1
			\end{bNiceMatrix}
			\begin{bNiceMatrix}
				d_{1} & 0     & 0     \\
				0     & d_{2} & 0     \\
				0     & 0     & d_{3}
			\end{bNiceMatrix}
			{\begin{bNiceMatrix}
				 1         & 0         & 0 \\
				 \ell_{21} & 1         & 0 \\
				 \ell_{31} & \ell_{32} & 1
			 \end{bNiceMatrix}}^{T} \\
			\begin{bNiceMatrix}
				50 & 12 & 35 \\
				12 & 3  & 9  \\
				35 & 9  & 29
			\end{bNiceMatrix}
			 & =
			\begin{bNiceMatrix}
				1             & 0 & 0 \\
				\frac{12}{50} & 1 & 0 \\
				\frac{35}{50} & 5 & 1
			\end{bNiceMatrix}
			\begin{bNiceMatrix}
				50 & 0            & 0             \\
				0  & \frac{6}{50} & 0             \\
				0  & 0            & \frac{75}{50}
			\end{bNiceMatrix}
			{\begin{bNiceMatrix}
				 1             & 0 & 0 \\
				 \frac{12}{50} & 1 & 0 \\
				 \frac{35}{50} & 5 & 1
			 \end{bNiceMatrix}}^{T}
		\end{align*}
	\end{solution}
\end{frame}

\begin{frame}
	\frametitle{Factorización $LL^{T}$ de Choleski}
	\begin{solution}
		La factorización $LL^{T}$ no necesariamente tiene números 1 en la
		diagonal de la matriz triangular inferior, por lo que necesitamos
		tener
		\begin{align*}
			\widetilde{A}
			 & =
			LL^{T}                             \\
			\begin{bNiceMatrix}
				a_{11} & a_{21} & a_{31} \\
				a_{21} & a_{22} & a_{32} \\
				a_{31} & a_{32} & a_{33}
			\end{bNiceMatrix}
			 & =
			\begin{bNiceMatrix}
				\ell_{11} & 0         & 0         \\
				\ell_{21} & \ell_{22} & 0         \\
				\ell_{31} & \ell_{32} & \ell_{33}
			\end{bNiceMatrix}
			{\begin{bNiceMatrix}
				 \ell_{11} & 0         & 0         \\
				 \ell_{21} & \ell_{22} & 0         \\
				 \ell_{31} & \ell_{32} & \ell_{33}
			 \end{bNiceMatrix}}^{T} \\
			\begin{bNiceMatrix}
				50 & 12 & 35 \\
				12 & 3  & 9  \\
				35 & 9  & 29
			\end{bNiceMatrix}
			 & =
			\begin{bNiceMatrix}
				\ell_{11}^{2}      & \ell_{11}\ell_{21}                    & \ell_{11}\ell_{31}                        \\
				\ell_{11}\ell_{21} & \ell_{21}^{2}+\ell_{22}^{2}           & \ell_{21}\ell_{31}+\ell_{22}\ell_{32}     \\
				\ell_{11}\ell_{31} & \ell_{21}\ell_{31}+\ell_{22}\ell_{32} & \ell_{31}^{2}+\ell_{32}^{2}+\ell_{33}^{2}
			\end{bNiceMatrix}
		\end{align*}
		Por lo tanto,

		\begin{align*}
			a_{11}:50=\ell_{11}^{2}                                             & \implies\ell_{11}=5\sqrt{2}                   &
			a_{21}:12=\alert{\ell_{11}}\ell_{21}                                & \implies\ell_{21}=\frac{12}{5\sqrt{2}}        & \\
			a_{31}:35=\alert{\ell_{11}}\ell_{31}                                & \implies\ell_{31}=\frac{35}{5\sqrt{2}}        &
			a_{22}:3=\alert{\ell_{21}}^{2}+\ell_{22}^{2}                        & \implies\ell_{22}=\frac{\sqrt{6}}{5\sqrt{2}}  & \\
			a_{32}:9=\alert{\ell_{21}}l_{31}+\alert{\ell_{22}}\ell_{32}         & \implies\ell_{32}=\frac{6}{\sqrt{12}}         &
			a_{33}:29=\alert{\ell_{31}}^{2}+\alert{\ell_{32}}^{2}+\ell_{33}^{2} & \implies\ell_{33}=\frac{\sqrt{15}}{\sqrt{10}} &
		\end{align*}
	\end{solution}
\end{frame}

\begin{frame}
	\begin{solution}
		Así, obtenemos
		\begin{align*}
			\widetilde{A}
			 & =
			LL^{T}                             \\
			\begin{bNiceMatrix}
				50 & 12 & 35 \\
				12 & 3  & 9  \\
				35 & 9  & 29
			\end{bNiceMatrix}
			 & =
			\begin{bNiceMatrix}
				\ell_{11} & 0         & 0         \\
				\ell_{21} & \ell_{22} & 0         \\
				\ell_{31} & \ell_{32} & \ell_{33}
			\end{bNiceMatrix}
			{\begin{bNiceMatrix}
				 \ell_{11} & 0         & 0         \\
				 \ell_{21} & \ell_{22} & 0         \\
				 \ell_{31} & \ell_{32} & \ell_{33}
			 \end{bNiceMatrix}}^{T} \\
			\begin{bNiceMatrix}
				50 & 12 & 35 \\
				12 & 3  & 9  \\
				35 & 9  & 29
			\end{bNiceMatrix}
			 & =
			\begin{bNiceMatrix}
				5\sqrt{2}            & 0                          & 0                           \\
				\frac{12}{5\sqrt{2}} & \frac{\sqrt{6}}{5\sqrt{2}} & 0                           \\
				\frac{35}{5\sqrt{2}} & \frac{6}{\sqrt{12}}        & \frac{\sqrt{15}}{\sqrt{10}}
			\end{bNiceMatrix}
			{\begin{bNiceMatrix}
				 5\sqrt{2}            & 0                          & 0                           \\
				 \frac{12}{5\sqrt{2}} & \frac{\sqrt{6}}{5\sqrt{2}} & 0                           \\
				 \frac{35}{5\sqrt{2}} & \frac{6}{\sqrt{12}}        & \frac{\sqrt{15}}{\sqrt{10}}
			 \end{bNiceMatrix}}^{T}
		\end{align*}
		Verificaciones de las factorizaciones $LDL^{T}$ y $LL^{T}$ de
		Choleski.

		\begin{minipage}{0.65\textwidth}
			\begin{listing}[H]
				\inputminted[
					fontsize=\scriptsize,
					breaklines,
					firstline=8,
					lastline=23
				]{python}{pregunta7.py}
			\end{listing}
		\end{minipage}
		\begin{minipage}{0.25\textwidth}
			\begin{listing}[H]
				\inputminted[
					fontsize=\scriptsize,
					firstline=1,
					lastline=12
				]{text}{output_pregunta7.txt}
			\end{listing}
		\end{minipage}
	\end{solution}
\end{frame}

\begin{frame}
	\frametitle{Implementación en Python}
	\begin{solution}
		\begin{minipage}{0.45\textwidth}
			\begin{listing}[H]
				\inputminted[
					fontsize=\tiny,
					breaklines,
					firstline=5,
					lastline=38
				]{python}{ldlt.py}
			\end{listing}
		\end{minipage}
		\begin{minipage}{0.45\textwidth}
			\begin{listing}[H]
				\inputminted[
					fontsize=\tiny,
					breaklines,
					firstline=4,
					lastline=20
				]{python}{choleskyllt.py}
			\end{listing}
		\end{minipage}
	\end{solution}
\end{frame}

\begin{frame}[fragile]
	\begin{enumerate}\setcounter{enumi}{15}
		\item

		      Dado la matriz
		      \begin{math}
			      A=
			      \begin{bNiceMatrix}
				      1 & 1 & 1 \\
				      1 & 1 & 1 \\
				      1 & 1 & 1
			      \end{bNiceMatrix}
		      \end{math},
		      determine la descomposición Parlett-Reid.
	\end{enumerate}

	\begin{solution}
		Notamos que la matriz simétrica $A$ tiene autovalores
		$\lambda_{1}=3$, $\lambda_{2}=\lambda_{3}=0$.
		Es decir, $A$ es una matriz semidefinida positiva.
		Por el método de Parlett-Reid, se cumple $PAP^{t}=LTL^{t}$,
		el primer paso es
		\begin{equation*}
			P_{1}=
			\begin{bNiceMatrix}
				e_{1} & e_{3} & e_{2}
			\end{bNiceMatrix}=
			\begin{bNiceMatrix}
				1 & 0 & 0 \\
				0 & 0 & 1 \\
				0 & 1 & 0
			\end{bNiceMatrix},\quad
			P_1^{t}=
			\begin{bNiceMatrix}
				1 & 0 & 0 \\
				0 & 0 & 1 \\
				0 & 1 & 0
			\end{bNiceMatrix},\quad
			\alpha_{1}=
			\begin{bNiceMatrix}
				0 \\
				0 \\
				1
			\end{bNiceMatrix},\quad
			M_{1}^{t}=
			\begin{bNiceMatrix}
				1 & 0 & 0  \\
				0 & 1 & -1 \\
				0 & 0 & 1
			\end{bNiceMatrix}.
		\end{equation*}

		\begin{equation*}
			M_{1}=
			I_{3}-\alpha_{1}e_2^{t}=
			\begin{bNiceMatrix}
				1 & 0 & 0 \\
				0 & 1 & 0 \\
				0 & 0 & 1
			\end{bNiceMatrix}-
			\begin{bNiceMatrix}
				0 \\
				0 \\
				1
			\end{bNiceMatrix}
			\begin{bNiceMatrix}
				0 & 1 & 0
			\end{bNiceMatrix}=
			\begin{bNiceMatrix}
				1 & 0 & 0 \\
				0 & 1 & 0 \\
				0 & 0 & 1
			\end{bNiceMatrix}-
			\begin{bNiceMatrix}
				0 & 0 & 0 \\
				0 & 0 & 0 \\
				0 & 1 & 0
			\end{bNiceMatrix}=
			\begin{bNiceMatrix}
				1 & 0  & 0 \\
				0 & 1  & 0 \\
				0 & -1 & 1
			\end{bNiceMatrix}.
		\end{equation*}

		\begin{align*}
			A^{(1)}=
			M_{1}P_{1}AP_{1}^{t}M_{1}^{t}
			 & =
			\begin{bNiceMatrix}
				1 & 0  & 0 \\
				0 & 1  & 0 \\
				0 & -1 & 1
			\end{bNiceMatrix}
			\begin{bNiceMatrix}
				1 & 0 & 0 \\
				0 & 0 & 1 \\
				0 & 1 & 0
			\end{bNiceMatrix}
			\begin{bNiceMatrix}
				1 & 1 & 1 \\
				1 & 1 & 1 \\
				1 & 1 & 1
			\end{bNiceMatrix}
			\begin{bNiceMatrix}
				1 & 0 & 0 \\
				0 & 0 & 1 \\
				0 & 1 & 0
			\end{bNiceMatrix}
			\begin{bNiceMatrix}
				1 & 0 & 0  \\
				0 & 1 & -1 \\
				0 & 0 & 1
			\end{bNiceMatrix} \\
			 & =
			\begin{bNiceMatrix}
				1 & 0 & 0  \\
				0 & 0 & 1  \\
				0 & 1 & -1
			\end{bNiceMatrix}
			\begin{bNiceMatrix}
				1 & 1 & 1 \\
				1 & 1 & 1 \\
				1 & 1 & 1
			\end{bNiceMatrix}
			\begin{bNiceMatrix}
				1 & 0 & 0  \\
				0 & 0 & 1  \\
				0 & 1 & -1
			\end{bNiceMatrix} \\
			 & =
			\begin{bNiceMatrix}
				1 & 1 & 1 \\
				1 & 1 & 1 \\
				0 & 0 & 0
			\end{bNiceMatrix}
			\begin{bNiceMatrix}
				1 & 0 & 0  \\
				0 & 0 & 1  \\
				0 & 1 & -1
			\end{bNiceMatrix}=
			\begin{bNiceMatrix}
				1 & 1 & 0 \\
				1 & 1 & 0 \\
				0 & 0 & 0
			\end{bNiceMatrix}=T,\quad
			P=P_{1},\quad
			L={\left(M_{1}P_{1}P^{t}\right)}^{-1}=
			\begin{bNiceMatrix}
				1 & 0 & 0 \\
				0 & 1 & 0 \\
				0 & 1 & 1
			\end{bNiceMatrix}.
		\end{align*}
	\end{solution}
\end{frame}

\begin{frame}[fragile]
	\begin{minipage}{0.45\textwidth}
		\begin{listing}[H]
			\inputminted[
				fontsize=\scriptsize,
				breaklines,
				firstline=30,
				lastline=64
			]{python}{parlett-reid-edward.py}
		\end{listing}
	\end{minipage}
	\begin{minipage}{0.45\textwidth}
		\begin{listing}[H]
			\inputminted[
				fontsize=\scriptsize,
				breaklines,
				firstline=4,
				lastline=27
			]{python}{parlett-reid-edward.py}
			\inputminted[
				fontsize=\scriptsize,
				firstline=7,
				lastline=22
			]{text}{output_parlett.txt}
		\end{listing}
	\end{minipage}
\end{frame}

\begin{frame}
	\begin{enumerate}\setcounter{enumi}{16}
		\item

		      Dado la matriz
		      \begin{math}
			      A=
			      \begin{bNiceMatrix}
				      0 & 1 & 0 & 0 \\
				      0 & 0 & 2 & 0 \\
				      0 & 0 & 0 & 3 \\
				      0 & 0 & 0 & 0
			      \end{bNiceMatrix}
		      \end{math},
		      determine la descomposición SVD.
	\end{enumerate}

	\begin{solution}
		Sea $A=U\Sigma V^{\ast}$.
		\begin{math}
			A^{t}=
			\begin{bNiceMatrix}
				0 & 0 & 0 & 0 \\
				1 & 0 & 0 & 0 \\
				0 & 2 & 0 & 0 \\
				0 & 0 & 3 & 0
			\end{bNiceMatrix}
		\end{math}.
		Así
		\begin{math}
			\overline{A}=AA^{t}=
			\begin{bNiceMatrix}
				1 & 0 & 0 & 0 \\
				0 & 4 & 0 & 0 \\
				0 & 0 & 9 & 0 \\
				0 & 0 & 0 & 0
			\end{bNiceMatrix}
		\end{math}.

		El polinomio característico de $\overline{A}$ es
		\begin{equation*}
			p_{\overline{A}}\left(\lambda\right)=
			\det\left(\overline{A}-\lambda I\right)=
			-\lambda\left(1-\lambda\right)
			\left(4-\lambda\right)\left(9-\lambda\right),
		\end{equation*}
		cuyas raíces son 0, 1, 4 y 9.
		Si $\sigma^{2}_{1}=1$, $\sigma^{2}_{2}=4$, $\sigma^{2}_{3}=9$,
		entonces $\sigma_{1}=1$, $\sigma_{2}=2$, $\sigma_{3}=3$,
		$\sigma_{4}=0$.

		Luego,
		\begin{math}
			\Sigma=
			\begin{bNiceMatrix}
				\sigma_{1} & 0          & 0          & 0          \\
				0          & \sigma_{2} & 0          & 0          \\
				0          & 0          & \sigma_{3} & 0          \\
				0          & 0          & 0          & \sigma_{4}
			\end{bNiceMatrix}=
			\begin{bNiceMatrix}
				1 & 0 & 0 & 0 \\
				0 & 2 & 0 & 0 \\
				0 & 0 & 3 & 0 \\
				0 & 0 & 0 & 0
			\end{bNiceMatrix}
		\end{math}.

		Los autovectores de $\Sigma$ son
		\begin{equation*}
			u_{1}=
			\begin{bNiceMatrix}
				1 \\
				0 \\
				0 \\
				0
			\end{bNiceMatrix},\quad
			u_{2}=
			\begin{bNiceMatrix}
				0 \\
				1 \\
				0 \\
				0
			\end{bNiceMatrix},\quad
			u_{3}=
			\begin{bNiceMatrix}
				0 \\
				0 \\
				1 \\
				0
			\end{bNiceMatrix},\quad
			u_{4}=
			\begin{bNiceMatrix}
				0 \\
				0 \\
				0 \\
				1
			\end{bNiceMatrix},\quad
			U=
			\begin{bNiceMatrix}
				u_{1} & u_{2} & u_{3} & u_{4}
			\end{bNiceMatrix}=
			\begin{bNiceMatrix}
				1 & 0 & 0 & 0 \\
				0 & 1 & 0 & 0 \\
				0 & 0 & 1 & 0 \\
				0 & 0 & 0 & 1
			\end{bNiceMatrix}.
		\end{equation*}
	\end{solution}
\end{frame}

\begin{frame}
	\begin{solution}
		\begin{itemize}
			\item
			      \begin{math}
				      v_{1}=\frac{1}{\sigma_{1}}A^{t}u_{1}=
				      \frac{1}{1}\begin{bNiceMatrix}
					      0 & 0 & 0 & 0 \\
					      1 & 0 & 0 & 0 \\
					      0 & 2 & 0 & 0 \\
					      0 & 0 & 3 & 0
				      \end{bNiceMatrix}
				      \begin{bNiceMatrix}
					      1 \\
					      0 \\
					      0 \\
					      0
				      \end{bNiceMatrix}=
				      \begin{bNiceMatrix}
					      0 \\
					      1 \\
					      0 \\
					      0
				      \end{bNiceMatrix}
			      \end{math}.

			\item

			      \begin{math}
				      v_{2}=\frac{1}{\sigma_{2}}A^{t}u_{2}=
				      \frac{1}{2}
				      \begin{bNiceMatrix}
					      0 & 0 & 0 & 0 \\
					      1 & 0 & 0 & 0 \\
					      0 & 2 & 0 & 0 \\
					      0 & 0 & 0 & 3
				      \end{bNiceMatrix}
				      \begin{bNiceMatrix}
					      0 \\
					      1 \\
					      0 \\
					      0
				      \end{bNiceMatrix}=
				      \begin{bNiceMatrix}
					      0 \\
					      0 \\
					      1 \\
					      0
				      \end{bNiceMatrix}
			      \end{math}.

			\item

			      \begin{math}
				      v_{3}=\frac{1}{\sigma_{3}}A^{t}u_{3}=
				      \frac{1}{3}
				      \begin{bNiceMatrix}
					      0 & 0 & 0 & 0 \\
					      1 & 0 & 0 & 0 \\
					      0 & 2 & 0 & 0 \\
					      0 & 0 & 3 & 0
				      \end{bNiceMatrix}
				      \begin{bNiceMatrix}
					      0 \\
					      0 \\
					      1 \\
					      0
				      \end{bNiceMatrix}=
				      \begin{bNiceMatrix}
					      0 \\
					      0 \\
					      0 \\
					      1
				      \end{bNiceMatrix}
			      \end{math}.

			\item

			      \begin{math}
				      v_{4}=
				      \begin{bNiceMatrix}
					      1 \\
					      0 \\
					      0 \\
					      0
				      \end{bNiceMatrix}
			      \end{math}.
			\item

			      \begin{math}
				      V^{\ast}=
				      \begin{bNiceMatrix}
					      0 & 0 & 0 & 1 \\
					      1 & 0 & 0 & 0 \\
					      0 & 1 & 0 & 0 \\
					      0 & 0 & 1 & 0
				      \end{bNiceMatrix}
			      \end{math}.
		\end{itemize}
	\end{solution}
\end{frame}

\begin{frame}
	\begin{solution}
		\begin{minipage}{0.45\textwidth}
			\begin{listing}[H]
				\inputminted[
					fontsize=\scriptsize,
					breaklines,
					firstline=2,
					lastline=12
				]{python}{svd_alexandra.py}
				\inputminted[
					fontsize=\footnotesize,
				]{text}{svd_alexandra.txt}
			\end{listing}
		\end{minipage}
		\begin{minipage}{0.45\textwidth}
			\begin{listing}[H]
				\inputminted[
					fontsize=\scriptsize,
					breaklines,
					firstline=6,
					lastline=31
				]{python}{SVD.py}
			\end{listing}
		\end{minipage}
	\end{solution}
\end{frame}

\begin{frame}\transblindsvertical
	\frametitle{Referencias}
	%------------------------------------------------------------ 1
	\only<1>{
		\begin{itemize}
			\item Libros
			      \nocite{*}
			      \printbibliography[heading=none,keyword=book]
		\end{itemize}
	}
	%------------------------------------------------------------ 2
	\only<2>{
		\begin{itemize}
			\item Artículos científicos
			      \printbibliography[heading=none,keyword=paper]
		\end{itemize}
	}
	%------------------------------------------------------------ 3
	\only<3>{
		\begin{itemize}
			\item Sitios web
			      \printbibliography[heading=none,keyword=online]
		\end{itemize}
	}
\end{frame}

\end{document}