%! Aldo Luna, Alexandra Mazzetti, Carlos Aznarán, Edward Canales
%! Universidad Nacional de Ingeniería
%! Facultad de Ciencias
%! Lima, Perú
%! Uso:
%! $ sudo pacman -Syu texlive-most zathura # dependencias, visor
%! $ arara solution_tercera
%! $ zathura solution_tercera.pdf
%! Ver https://wiki.archlinux.org/title/TeX_Live
% arara: clean: {
% arara: --> extensions:
% arara: --> ['aux','log','nav',
% arara: --> 'out','snm','toc','pytxcode','pdf']
% arara: --> }
% arara: lualatex: {
% arara: --> shell: yes,
% arara: --> draft: yes,
% arara: --> interaction: batchmode
% arara: --> }
% arara: biber
%! arara: pythontex
% arara: lualatex: {
% arara: --> shell: yes,
% arara: --> draft: yes,
% arara: --> interaction: batchmode
% arara: --> }
% arara: lualatex: {
% arara: --> shell: yes,
% arara: --> synctex: yes,
% arara: --> interaction: batchmode
% arara: --> }
% arara: clean: {
% arara: --> extensions:
% arara: --> ['aux','log','nav',
% arara: --> 'out','snm','toc','pytxcode']
% arara: --> }
\PassOptionsToPackage{svgnames}{xcolor}
\documentclass[
	spanish,
	8pt,
	utf8,
	xcolor=table,
	handout,
	aspectratio=169,
	professionalfonts,
	% notheorems,
	mathserif,
	leqno,
	% t
]{beamer}
\setbeamersize{text margin left=5pt,text margin right=5pt}
\usepackage[spanish,es-sloppy]{babel}
\spanishdatedel\decimalpoint
\usepackage{mathtools}
\usepackage{nicematrix}
\usepackage{systeme}
\usepackage{minted}
\usepackage{enumerate}
\usepackage{multicol}
% \usepackage{pythontex}
\usepackage[
	citestyle=numeric,
	style=apa,
	backend=biber,
	defernumbers=true,
	sorting=ynt,
	maxcitenames=4
]{biblatex}
\addbibresource{references.bib}

\newcolumntype{x}[1]{>{\centering\arraybackslash\hspace{0pt}}p{#1}}

\newcounter{savedenum}
\newcommand*{\saveenum}{\setcounter{savedenum}{\theenumi}}
\newcommand*{\resume}{\setcounter{enumi}{\thesavedenum}}

\setbeamertemplate{navigation symbols}{}
\setbeamertemplate{footline}{}
\setbeamertemplate{headline}{}

% https://tex.stackexchange.com/questions/68080/beamer-bibliography-icon
\setbeamertemplate{bibliography item}{%
	\ifboolexpr{ test {\ifentrytype{book}} or test {\ifentrytype{mvbook}}
		or test {\ifentrytype{collection}} or test {\ifentrytype{mvcollection}}
		or test {\ifentrytype{reference}} or test {\ifentrytype{mvreference}} }
	{\setbeamertemplate{bibliography item}[book]}
	{\ifentrytype{online}
		{\setbeamertemplate{bibliography item}[online]}
		{\setbeamertemplate{bibliography item}[article]}}%
	\usebeamertemplate{bibliography item}}

\defbibenvironment{bibliography}
{\list{}
	{\settowidth{\labelwidth}{\usebeamertemplate{bibliography item}}%
		\setlength{\leftmargin}{\labelwidth}%
		\setlength{\labelsep}{\biblabelsep}%
		\addtolength{\leftmargin}{\labelsep}%
		\setlength{\itemsep}{\bibitemsep}%
		\setlength{\parsep}{\bibparsep}}}
{\endlist}
{\item}

\makeatletter
\newenvironment<>{proofs}[1][\proofname]{%
    \par
    \def\insertproofname{#1\@addpunct{.}}%
    \usebeamertemplate{proof begin}#2}
  {\usebeamertemplate{proof end}}
\makeatother


\title{
	\huge\sffamily
	Tercera Práctica Dirigida\quad Grupo N$^{\circ}$~1
}

\subtitle{
	\large\scshape
	Análisis y Modelamiento Numérico I\quad CM4F1 B\\[.5\baselineskip]
		\normalsize\normalfont
		Profesor Jonathan Munguia La Cotera
}

\author{
	Aldo Luna Bueno\quad\and\quad
	Alexandra Gutierrez Mazzetti\quad\and\quad
	Carlos Aznarán Laos\quad\and\quad
	Edward Canales Yarin
}

\institute{\large
	Facultad de Ciencias \and
	Universidad Nacional de Ingeniería
}
\date{26 de octubre del 2022}

\begin{document}

\frame{\titlepage}

\begin{frame}
	\frametitle{Factorizaciones $LU$}
	Suponga que $A$ se puede factorizar en el producto de una matriz
	triangular inferior $L$ y una matriz triangular superior $U$.
	Entonces para resolver el sistema de ecuaciones $A x=b$ basta con
	resolver este problema en dos etapas:
	\begin{equation*}
		Lz=b\text{ resolver para }z,\qquad
		Ux=z\text{ resolver para }x.
	\end{equation*}
	Comenzamos con una matriz $A$ de orden $n\times n$ y buscamos
	matrices
	\begin{equation*}
		L=
		\begin{bNiceMatrix}
			\ell_{11}  & 0          & 0          & \cdots & 0          \\
			\ell_{21}  & \ell_{22}  & 0          & \cdots & 0          \\
			\ell_{31}  & \ell_{32}  & \ell_{33}  & \cdots & 0          \\
			\vdots     & \vdots     & \vdots     & \ddots & \vdots     \\
			\ell_{n 1} & \ell_{n 2} & \ell_{n 3} & \cdots & \ell_{n n}
		\end{bNiceMatrix}\text{ y }
		U=
		\begin{bNiceMatrix}
			u_{11} & u_{12} & u_{13} & \cdots & u_{1 n} \\
			0      & u_{22} & u_{23} & \cdots & u_{2 n} \\
			0      & 0      & u_{33} & \cdots & u_{3 n} \\
			\vdots & \vdots & \vdots & \ddots & \vdots  \\
			0      & 0      & 0      & \cdots & u_{n n}
		\end{bNiceMatrix}
	\end{equation*}
	tal que $A=LU$.
	Cuando esto es posible, decimos que $A$ tiene descomposición $LU$.
	Resulta que $L$ y $U$ no son únicamente determinados.
	De hecho, para cada $i$, podemos asignar un valor no nulo o bien
	para $\ell_{ii}$ o bien para $u_{ii}$.
	Por ejemplo, una elección simple es asignar $\ell_{ii}=1$ para
	$i\in\left\{1,\ldots,n\right\}$, haciendo así $L$ una matriz
	triangular inferior unitaria.

	\begin{theorem}
		Si todos los $n$ menores principales líderes de una matriz $A$ de
		orden $n\times n$ son no singulares, entonces $A$ tiene una
		descomposición $LU$.
	\end{theorem}
\end{frame}

\begin{frame}
	\begin{proofs}
		Recordemos que el $k$-ésimo menor principal líder de la matriz
		$A$ es la matriz
		\begin{equation*}
			A_{k}=
			\begin{bNiceMatrix}
				a_{11}  & a_{12}  & \cdots & a_{1 k} \\
				a_{21}  & a_{22}  & \cdots & a_{2 k} \\
				\vdots  & \vdots  & \ddots & \vdots  \\
				a_{k 1} & a_{k 2} & \cdots & a_{k k}
			\end{bNiceMatrix}
		\end{equation*}
		Sean $A_{k}$, $L_{k}$, y $U_{k}$ el $k$-ésimo menor principal
		líder de las matrices $A, L$, y $U$, respectivamente.
		Nuestra hipótesis es que $A_{1},A_{2},\ldots,A_{n}$ son no
		singulares.

		A los efectos de una demostración inductiva, suponga que $L_{k-1}$
		y $U_{k-1}$ se han obtenido.
		Si $i,j\in\left\{1,\ldots,k-1\right\}$, entonces $s$ también está.
		Por eso,
		\begin{equation*}
			A_{k-1}=
			L_{k-1}U_{k-1}.
		\end{equation*}
		Ya que $A_{k-1}$ es no singular por hipótesis, $L_{k-1}$ y
		$U_{k-1}$ son también no singulares.
		Dado que $L_{k-1}$ es no singular, podemos resolver el sistema
		\begin{equation*}
			\sum_{s=1}^{k-1}
			\ell_{is}u_{sk}=
			a_{\imath k},\quad i\in\left\{1,\ldots,k-1\right\}
		\end{equation*}
		para $u_{sk}$ con $s\in\left\{1,\ldots,k-1\right\}$.
	\end{proofs}
\end{frame}

\begin{frame}
	\begin{proof}[\proofname\ (Cont.)]
		Estos elementos caen en la $k$-ésima columna de $U$.
		Dado que $U_{k-1}$ es no singular, podemos resolver el sistema
		\begin{equation*}
			\sum_{s=1}^{k-1}
			\ell_{ks}u_{si}=
			a_{kj},\quad j\in\left\{1,\ldots,k-1\right\}
		\end{equation*}
		para $\ell_{ks}$ con $s\in\left\{1,\ldots,k-1\right\}$.
		Estos elementos caen en la $k$-ésima fila de $L$.
		From the requirement
		\begin{equation*}
			a_{kk}=
			\sum_{s=1}^{k}\ell_{ks}u_{s k}=
			\sum_{s=1}^{k-1}
			\ell_{k s}u_{s k}+\ell_{kk}u_{kk}
		\end{equation*}
		podemos obtener $u_{kk}$ dado que $\ell_{kk}$ ha sido considerado como unidad.
		Thus, all the new elements necessary to form $L_k$ and $U_k$ have been defined.
		The induction is completed by noting that $\ell_{11} u_{11}=a_{11}$ and, therefore, $\ell_{11}=1$ and $u_{11}=a_{11}$.
	\end{proof}
\end{frame}

\begin{frame}

	\begin{enumerate}\setcounter{enumi}{1}
		\item

		      Considere la matriz
		      \begin{math}
			      \begin{bNiceMatrix}
				      2 & 2 & 1 \\
				      1 & 1 & 1 \\
				      3 & 2 & 1
			      \end{bNiceMatrix}
		      \end{math}.

		      \begin{enumerate}[a)]

			      \item

			            Demuestre que $A$ no se puede factorizar en el
			            producto de una matriz triangular inferior
			            unitaria y una matriz triangular superior.

			      \item

			            Intercambie las filas de $A$ para que esto se pueda
			            hacer.
		      \end{enumerate}

	\end{enumerate}

	\begin{solution}
		\begin{enumerate}[a)]

			\item

			      La matriz $A$ no se puede factorizar por el método LU porque $\exists k_{0}=2\in\left\{1,2,3\right\}$ tal que
			      $\det\left(A_{2}\right)=\begin{vNiceMatrix}
					      2 & 2 \\
					      1 & 1 \\
				      \end{vNiceMatrix}=0$.

			\item Intercambiando la fila 2 con la fila 3 obtenemos $\widetilde{A}=\begin{bNiceMatrix}
					      2 & 2 & 1 \\
					      3 & 2 & 1 \\
					      1 & 1 & 1
				      \end{bNiceMatrix}$

			      .
		\end{enumerate}
	\end{solution}
\end{frame}

\begin{frame}[fragile]
	\begin{minipage}{0.45\textwidth}
		\begin{listing}[H]
			\inputminted[
				fontsize=\scriptsize,
				breaklines,
				firstline=5,
				lastline=20
			]{python}{lualdo.py}
			\inputminted[
				fontsize=\footnotesize,
			]{text}{output.txt}
		\end{listing}
	\end{minipage}
	\begin{minipage}{0.45\textwidth}
		\begin{listing}[H]
			\inputminted[
				fontsize=\scriptsize,
				breaklines,
				firstline=23,
				lastline=51
			]{python}{lualdo.py}
		\end{listing}
	\end{minipage}
\end{frame}

\begin{frame}
	\begin{enumerate}\setcounter{enumi}{6}
		\item

		      Una compañía minera trabaja en 3 minas, cada una de las
		      cuales produce minerles de tres clases.
		      La primera mina puede producir 4 toneladas del mineral A, 3
		      toneladas del mineral B, y 5 toneladas del mineral C; la
		      segunda mina puede producir 1 tonelada de cada uno de los
		      minerales y la tercera mina, 2 toneladas del A,
		      4 toneladas del B y 3 toneladas del C, por cada hora de
		      funcionamiento.
		      Se desea satisfacer los tres pedidos siguientes
		      \begin{table}[ht!]
			      \centering
			      \tiny
			      \begin{tabular}{|c|c|c|c|}
				      \hline
				      Pedidos & Mineral A & Mineral B & Mineral C \\
				      \hline
				      $P_{1}$ & $19$      & $25$      & $25$      \\
				      \hline
				      $P_{2}$ & $13$      & $16$      & $16$      \\
				      \hline
				      $P_{3}$ & $8$       & $12$      & $10$      \\
				      \hline
			      \end{tabular}
		      \end{table}

		      Determine
		      \begin{multicols}{2}

			      \begin{enumerate}[a)]
				      \item

				            Modele el sistema a resolver.

				      \item

				            Resolver usando los programas $LDL^{T}$ y Cholesky.
			      \end{enumerate}
		      \end{multicols}

	\end{enumerate}

	\begin{solution}
		\begin{enumerate}[a)]

			\item

			      .
		\end{enumerate}
	\end{solution}
\end{frame}

\begin{frame}
	\begin{enumerate}\setcounter{enumi}{15}
		\item

		      Dado la matriz
		      \begin{math}
			      A=
			      \begin{bNiceMatrix}
				      1 & 1 & 1 \\
				      1 & 1 & 1 \\
				      1 & 1 & 1
			      \end{bNiceMatrix}
		      \end{math},
		      determine la descomposición Parlett-Reid.
	\end{enumerate}

	\begin{solution}

		Notamos que el $\det(A)=0$.
		$A^{t}=A$, es decir, $A$ es simétrica.
		Y además, $A$ es semidefinida positiva porque tenemos como autovalores a $\lambda_{1}=3$, $\lambda_{2}=\lambda_{3}=0$.

		$\det(A_{1})=1$, $\det(A_{2})=0$, $\det(A_{3})=0$
		% \begin{aligned}

		% 	&P_1=\left[e_1, e_3, e_2\right]=\left[\begin{array}{lll}
		% 	1 & 0 & 0 \\
		% 	0 & 0 & 1 \\
		% 	0 & 1 & 0
		% 	\end{array}\right] \\
		% 	&\alpha_1=\left[\begin{array}{l}
		% 	0 \\
		% 	0 \\
		% 	1 / 1
		% 	\end{array}\right]=\left[\begin{array}{l}
		% 	0 \\
		% 	0 \\
		% 	1
		% 	\end{array}\right]
		% 	\end{aligned}
	\end{solution}
\end{frame}

\begin{frame}[fragile]
	\begin{minipage}{0.45\textwidth}
		\begin{listing}[H]
			\inputminted[
				fontsize=\scriptsize,
				breaklines,
				firstline=30,
				lastline=64
			]{python}{parlett-reid-edward.py}
		\end{listing}
	\end{minipage}
	\begin{minipage}{0.45\textwidth}
		\begin{listing}[H]
			\inputminted[
				fontsize=\scriptsize,
				breaklines,
				firstline=4,
				lastline=27
			]{python}{parlett-reid-edward.py}
			% \inputminted[
			% 	fontsize=\scriptsize,
			% ]{text}{output_parlett.txt}
		\end{listing}
	\end{minipage}
\end{frame}

\begin{frame}
	\begin{enumerate}\setcounter{enumi}{16}
		\item

		      Dado la matriz
		      \begin{math}
			      A=
			      \begin{bNiceMatrix}
				      0 & 1 & 0 & 0 \\
				      0 & 0 & 2 & 0 \\
				      0 & 0 & 0 & 3 \\
				      0 & 0 & 0 & 0
			      \end{bNiceMatrix}
		      \end{math},
		      determine la descomposición SVD.
	\end{enumerate}

	\begin{solution}
	\end{solution}
\end{frame}

\begin{frame}\transblindsvertical
	\frametitle{Referencias}
	%------------------------------------------------------------ 1
	\only<1>{
		\begin{itemize}
			\item Libros
			      \nocite{*}
			      \printbibliography[heading=none,keyword=book]
		\end{itemize}
	}
	%------------------------------------------------------------ 2
	\only<2>{
		\begin{itemize}
			\item Artículos científicos
			      \printbibliography[heading=none,keyword=paper]
		\end{itemize}
	}
	%------------------------------------------------------------ 3
	\only<3>{
		\begin{itemize}
			\item Sitios web
			      \printbibliography[heading=none,keyword=online]
		\end{itemize}
	}
\end{frame}

\end{document}