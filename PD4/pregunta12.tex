\begin{frame}

	\begin{enumerate}\setcounter{enumi}{11}
		\item

		      Use los métodos de SOR y del descenso más rápido para
		      encontrar la solución de sistema $Ax=b$ con una precisión
		      de $10^{-5}$ en la norma ${\left\|\cdot\right\|}_{\infty}$.
		      % \setcounter{MaxMatrixCols}{30}
		      \begin{equation*}
			      a_{ij}=
			      \begin{cases}
				      4,  & \text {si }j=1\text { e }i=1,\ldots,16,      \\
				      -1, & \text {si }
				      \begin{cases}
					      j=i+1 \text { e } i=1,2,3,5,6,7,9,10,11,13,14,15,  \\
					      j=i-1 \text { e } i=2,3,4,6,7,8,10,11,12,14,15,16, \\
					      j=i+4 \text { e } i=1,\ldots, 12,                  \\
					      j=i-4 \text { e } i=5,\ldots, 16,
				      \end{cases} \\
				      0,  & \text {en otro caso}.
			      \end{cases}\quad
			      \text{y}\quad
			      b=
			      \tiny
			      \begin{bNiceMatrix}
				      1.902207  \\
				      1.051143  \\
				      1.175689  \\
				      3.480083  \\
				      0.819600  \\
				      -0.264419 \\
				      -0.412789 \\
				      1.175689  \\
				      0.913337  \\
				      -0.150209 \\
				      -0.264419 \\
				      1.051143  \\
				      1.966694  \\
				      0.913337  \\
				      0.819600  \\
				      1.902207
			      \end{bNiceMatrix}
		      \end{equation*}
	\end{enumerate}
	\begin{solution}
		Sea
		\begin{math}\tiny
			A =
			\setcounter{MaxMatrixCols}{25}
			\begin{bNiceMatrix}
				4 & -1 & 0  & 0  & -1 & 0  & 0  & 0  & 0  & 0  & 0  & 0  & 0  & 0  & 0  & 0  \\
				4 & 0  & -1 & 0  & 0  & -1 & 0  & 0  & 0  & 0  & 0  & 0  & 0  & 0  & 0  & 0  \\
				4 & -1 & 0  & -1 & 0  & 0  & -1 & 0  & 0  & 0  & 0  & 0  & 0  & 0  & 0  & 0  \\
				4 & 0  & -1 & 0  & 0  & 0  & 0  & -1 & 0  & 0  & 0  & 0  & 0  & 0  & 0  & 0  \\
				4 & 0  & 0  & 0  & 0  & -1 & 0  & 0  & -1 & 0  & 0  & 0  & 0  & 0  & 0  & 0  \\
				4 & -1 & 0  & 0  & -1 & 0  & -1 & 0  & 0  & -1 & 0  & 0  & 0  & 0  & 0  & 0  \\
				4 & 0  & -1 & 0  & 0  & -1 & 0  & -1 & 0  & 0  & -1 & 0  & 0  & 0  & 0  & 0  \\
				4 & 0  & 0  & -1 & 0  & 0  & -1 & 0  & 0  & 0  & 0  & -1 & 0  & 0  & 0  & 0  \\
				4 & 0  & 0  & 0  & -1 & 0  & 0  & 0  & 0  & -1 & 0  & 0  & -1 & 0  & 0  & 0  \\
				4 & 0  & 0  & 0  & 0  & -1 & 0  & 0  & -1 & 0  & -1 & 0  & 0  & -1 & 0  & 0  \\
				4 & 0  & 0  & 0  & 0  & 0  & -1 & 0  & 0  & -1 & 0  & -1 & 0  & 0  & -1 & 0  \\
				4 & 0  & 0  & 0  & 0  & 0  & 0  & -1 & 0  & 0  & -1 & 0  & 0  & 0  & 0  & -1 \\
				4 & 0  & 0  & 0  & 0  & 0  & 0  & 0  & -1 & 0  & 0  & 0  & 0  & -1 & 0  & 0  \\
				4 & 0  & 0  & 0  & 0  & 0  & 0  & 0  & 0  & -1 & 0  & 0  & -1 & 0  & -1 & 0  \\
				4 & 0  & 0  & 0  & 0  & 0  & 0  & 0  & 0  & 0  & -1 & 0  & 0  & -1 & 0  & -1 \\
				4 & 0  & 0  & 0  & 0  & 0  & 0  & 0  & 0  & 0  & 0  & -1 & 0  & 0  & -1 & 0
			\end{bNiceMatrix}
		\end{math}
		pero \alert{no es simétrica} y
		\alert{no es definida positiva}.
	\end{solution}
\end{frame}

\begin{frame}
	\begin{solution}
		\begin{minipage}{0.4\textwidth}
			\begin{listing}[H]
				\inputminted[
					fontsize=\tiny,
					breaklines,
					firstline=1,
					lastline=21
				]{text}{resultado_pregunta12.txt}
			\end{listing}
		\end{minipage}
		\begin{minipage}{0.5\textwidth}
			\begin{listing}[H]
				\inputminted[
					fontsize=\scriptsize,
					breaklines,
					firstline=6,
					lastline=17
				]{python}{pregunta12.py}
				\inputminted[
					fontsize=\scriptsize,
					breaklines,
					firstline=20,
					lastline=41
				]{python}{pregunta12.py}
			\end{listing}
		\end{minipage}
	\end{solution}
\end{frame}

\begin{frame}
	\begin{solution}
		Por lo que vamos a multplicar por $A^{T}$ al sistema $Ax=b$, resultando
		\begin{align*}
			Ax             & =b             \\
			A^{T}Ax        & =A^{T}b        \\
			\widetilde{A}x & =\widetilde{b}
		\end{align*}

		\begin{minipage}{0.45\textwidth}
			\begin{listing}[H]
				\inputminted[
					fontsize=\tiny,
					breaklines,
					firstline=23,
					lastline=39
				]{text}{resultado_pregunta12.txt}
				\inputminted[
					fontsize=\tiny,
					breaklines,
					firstline=95,
					lastline=97
				]{text}{resultado_pregunta12.txt}
			\end{listing}
		\end{minipage}
		\begin{minipage}{0.45\textwidth}
			\inputminted[
				fontsize=\tiny,
				breaklines,
				firstline=41,
				lastline=57
			]{text}{resultado_pregunta12.txt}
		\end{minipage}
	\end{solution}
\end{frame}

\begin{frame}
	\begin{solution}
		\begin{minipage}{0.45\textwidth}
			\begin{listing}[H]
				\inputminted[
					fontsize=\tiny,
					breaklines,
					firstline=59,
					lastline=75
				]{text}{resultado_pregunta12.txt}
			\end{listing}
		\end{minipage}
		\begin{minipage}{0.45\textwidth}
			\inputminted[
				fontsize=\tiny,
				breaklines,
				firstline=77,
				lastline=94
			]{text}{resultado_pregunta12.txt}
		\end{minipage}
	\end{solution}
\end{frame}

\begin{frame}
	\tiny
	\begin{solution}
		\begin{listing}[H]
			\inputminted[
				fontsize=\tiny,
				breaklines,
				firstline=99,
				lastline=134
			]{text}{resultado_pregunta12.txt}
		\end{listing}
	\end{solution}
\end{frame}

\begin{frame}
	\tiny
	\begin{solution}
		\begin{listing}[H]
			\inputminted[
				fontsize=\tiny,
				breaklines,
				firstline=135,
				lastline=171
			]{text}{resultado_pregunta12.txt}
		\end{listing}
	\end{solution}
\end{frame}

\begin{frame}
	\tiny
	\begin{solution}
		\begin{listing}[H]
			\inputminted[
				fontsize=\tiny,
				breaklines,
				firstline=172,
				lastline=206
			]{text}{resultado_pregunta12.txt}
		\end{listing}
	\end{solution}
\end{frame}

\begin{frame}
	\tiny
	\begin{solution}
		\begin{listing}[H]
			\inputminted[
				fontsize=\tiny,
				breaklines,
				firstline=207,
				lastline=225
			]{text}{resultado_pregunta12.txt}
		\end{listing}
		Dado $x^{\left(0\right)}=0\in\mathbb{R}^{16}$, el
		\alert{método del descenso más rápido} después de $59$
		iteraciones a la solución exacta
		\begin{math}
			\begin{pNiceMatrix}
				1.05  \\
				2.14  \\
				0.37  \\
				-1.68 \\
				0.15  \\
				2.73  \\
				2.56  \\
				0.8   \\
				0.67  \\
				-0.39 \\
				0.27  \\
				2.14  \\
				3.52  \\
				1.09  \\
				0.15  \\
				2.04
			\end{pNiceMatrix}\in
			\mathbb{R}^{16}
		\end{math}.
	\end{solution}
\end{frame}