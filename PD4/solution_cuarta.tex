%! Aldo Luna, Alexandra Mazzetti, Carlos Aznarán, Edward Canales
%! Universidad Nacional de Ingeniería
%! Facultad de Ciencias
%! Lima, Perú
%! Uso:
%! $ sudo pacman -Syu texlive-most zathura # dependencias, visor
%! $ arara solution_cuarta
%! $ zathura solution_cuarta.pdf
%! Ver https://wiki.archlinux.org/title/TeX_Live
% arara: clean: {
% arara: --> extensions:
% arara: --> ['aux','log','nav',
% arara: --> 'out','snm','toc','pytxcode','pdf']
% arara: --> }
% arara: lualatex: {
% arara: --> shell: yes,
% arara: --> draft: yes,
% arara: --> interaction: batchmode
% arara: --> }
% arara: biber
%! arara: pythontex
% arara: lualatex: {
% arara: --> shell: yes,
% arara: --> draft: yes,
% arara: --> interaction: batchmode
% arara: --> }
% arara: lualatex: {
% arara: --> shell: yes,
% arara: --> synctex: yes,
% arara: --> interaction: batchmode
% arara: --> }
% arara: clean: {
% arara: --> extensions:
% arara: --> ['aux','log','nav',
% arara: --> 'out','snm','toc','pytxcode']
% arara: --> }
\PassOptionsToPackage{svgnames}{xcolor}
\documentclass[
	spanish,
	8pt,
	utf8,
	xcolor=table,
	handout,
	aspectratio=169,
	professionalfonts,
	% notheorems,
	mathserif,
	leqno,
	% t
]{beamer}
\setbeamersize{text margin left=5pt,text margin right=5pt}
\usepackage[spanish,es-sloppy]{babel}
\spanishdatedel\decimalpoint
\usepackage{mathtools}
\usepackage{nicematrix}
\usepackage{systeme}
\usepackage{minted}
\usepackage{enumerate}
\usepackage{multicol}
% \usepackage{pythontex}
\usepackage[
	citestyle=numeric,
	style=apa,
	backend=biber,
	defernumbers=true,
	sorting=ynt,
	maxcitenames=4
]{biblatex}
\addbibresource{references.bib}

\newcolumntype{x}[1]{>{\centering\arraybackslash\hspace{0pt}}p{#1}}

\newcounter{savedenum}
\newcommand*{\saveenum}{\setcounter{savedenum}{\theenumi}}
\newcommand*{\resume}{\setcounter{enumi}{\thesavedenum}}

\setbeamertemplate{navigation symbols}{}
\setbeamertemplate{footline}{}
\setbeamertemplate{headline}{}

% https://tex.stackexchange.com/questions/68080/beamer-bibliography-icon
\setbeamertemplate{bibliography item}{%
	\ifboolexpr{ test {\ifentrytype{book}} or test {\ifentrytype{mvbook}}
		or test {\ifentrytype{collection}} or test {\ifentrytype{mvcollection}}
		or test {\ifentrytype{reference}} or test {\ifentrytype{mvreference}} }
	{\setbeamertemplate{bibliography item}[book]}
	{\ifentrytype{online}
		{\setbeamertemplate{bibliography item}[online]}
		{\setbeamertemplate{bibliography item}[article]}}%
	\usebeamertemplate{bibliography item}}

\defbibenvironment{bibliography}
{\list{}
	{\settowidth{\labelwidth}{\usebeamertemplate{bibliography item}}%
		\setlength{\leftmargin}{\labelwidth}%
		\setlength{\labelsep}{\biblabelsep}%
		\addtolength{\leftmargin}{\labelsep}%
		\setlength{\itemsep}{\bibitemsep}%
		\setlength{\parsep}{\bibparsep}}}
{\endlist}
{\item}

\makeatletter
\newenvironment<>{proofs}[1][\proofname]{%
    \par
    \def\insertproofname{#1\@addpunct{.}}%
    \usebeamertemplate{proof begin}#2}
  {\usebeamertemplate{proof end}}
\makeatother


\title{
	\huge\sffamily
	Cuarta Práctica Dirigida\quad Grupo N$^{\circ}$~3
}

\subtitle{
	\large\scshape
	Análisis y Modelamiento Numérico I\quad CM4F1 B\\[.5\baselineskip]
		\normalsize\normalfont
		Profesor Jonathan Munguia La Cotera
}

\author{
	Aldo Luna Bueno\quad\and\quad
	Alexandra Gutierrez Mazzetti\quad\and\quad
	Carlos Aznarán Laos\quad\and\quad
	Edward Canales Yarin
}

\institute{\large
	Facultad de Ciencias \and
	Universidad Nacional de Ingeniería
}
\date{16 de noviembre del 2022}

\begin{document}

\frame{\titlepage}

\begin{frame}
	Para estudiar la convergencia de técnicas de iteración general, necesitamos analizar la fórmula
	\begin{equation*}
		\boxed{
			\forall k\geq0:
			x^{\left(k+1\right)}=
			Tx^{\left(k\right)}+c.
		}
	\end{equation*}

	\begin{definition}[Radio espectral]
		Sea $A\in\mathbb{C}^{n\times n}$ con autovalores $\lambda_{i}$,
		$1\leq n$.
		Entonces
		\begin{equation*}
			\rho\left(A\right)\coloneqq
			\max_{1\leq i\leq n}\left|\lambda\right|.
		\end{equation*}
		Geométricamente, si todos los autovalores $\lambda_{i}$ de $A$
		son ubicados en el $z$-plano complejo, entonces
		$\rho\left(A\right)$ es el radio del menor disco
		$\left|z\right|\leq R$, con centro en el origen, que incluye
		todos los autovalores de la matriz $A$.
	\end{definition}

	\begin{theorem}[]
		Para cualquier $x^{\left(0\right)}\in\mathbb{R}^{n}$, la sucesión
		\begin{math}
			{\left\{x^{\left(k\right)}\right\}}_{k\in\mathbb{N}}
		\end{math}
		definida por
		\begin{math}
			x^{\left(k+1\right)}=
			Tx^{\left(k\right)}+
			c
		\end{math}
		converge a la \alert{solución única} de $x=Tx+c$ sii
		$\rho\left(T\right)<1$.

		\begin{definition}[Matriz convergente]
			Sea $A\in\mathbb{C}^{n\times n}$.
			$A$ es \alert{convergente}
			(\alert{a cero}) sii la sucesión
			\begin{math}
				{\left\{A^{\left(k\right)}\right\}}_{k\in\mathbb{N}}
			\end{math}
			converge a la matriz nula $O\in\mathbb{C}^{n\times n}$, y de lo
			contrario es divergente.
		\end{definition}

		\begin{theorem}[Criterio de convergencia]
			Si $A\in\mathbb{C}^{n\times n}$, entonces $A$ es convergente sii
			$\rho\left(A\right)<1$.
		\end{theorem}
		% \begin{proof}
		% 	\begin{itemize}
		% 		\item[$\left(\Rightarrow\right)$]

		% 			Primero suponga que $\rho\left(T\right)<1$.
		% 			Entonces,
		% 			\begin{align*}
		% 				x^{\left(k\right)}                       & =
		% 				T\alert{x^{\left(k-1\right)}}+c                                                                       \\
		% 				                                         & =
		% 				T\left(T x^{\left(k-2\right)}+c\right)+c                                                              \\
		% 				                                         & =T^{2}x^{\left(k-2\right)}+\left(T+I\right)c               \\
		% 				                                         & \vdotswithin{=}                                            \\
		% 				                                         & =T^{k}x^{\left(0\right)}+\left(T^{k-1}+\cdots+T+I\right)c.
		% 				\shortintertext{Puesto que $\rho\left(T\right)<1$, implica que $T$ es convergente y}
		% 				\lim_{k\to\infty}T^{k}x^{\left(0\right)} & =0.
		% 			\end{align*}
		% 	\end{itemize}
		% \end{proof}
	\end{theorem}
\end{frame}

% \begin{frame}
% 	\begin{proof}
% 		\begin{equation*}
% 			\lim_{k\to\infty}
% 			x^{\left(k\right)}=
% 			\lim_{k\to\infty}
% 			T^{k}x^{\left(0\right)}+
% 			\left(\sum_{j=0}^{\infty}T^{j}\right)c=
% 			0+{\left(I-T\right)}^{-1}c=
% 				{\left(I-T\right)}^{-1}c
% 		\end{equation*}
% 		Por lo tanto, la sucesión $\left\{\mathbf{x}^{(k)}\right\}$ converge al vector $\mathbf{x} \equiv(I-T)^{-1} \mathbf{c}$ y $\mathbf{x}=T \mathbf{x}+\mathbf{c}$.
% 		Para probar lo contrario, mostraremos que para cualquier $\mathbf{z} \in \mathbb{R}^n$, tenemos $\operatorname{lím}_{k \rightarrow \infty}$ $T^k \mathbf{z}=\mathbf{0}$. Con el teorema 7.17, esto es equivalente a $\rho(T)<1$.

% 		Sea $\mathbf{z}$ un vector arbitrario y $x$ la única solución para $\mathbf{x}=T \mathbf{x}+\mathbf{c}$. Defina $\mathbf{x}^{(0)}=\mathbf{x}-\mathbf{z}$, y, para $k \geq 1, \mathbf{x}^{(k)}=T \mathbf{x}^{(k-1)}+\mathbf{c}$. Entonces $\left\{\mathbf{x}^{(k)}\right\}$ converge a $\mathbf{x}$. También,
% 		$$
% 			\mathbf{x}-\mathbf{x}^{(k)}=(T \mathbf{x}+\mathbf{c})-\left(T \mathbf{x}^{(k-1)}+\mathbf{c}\right)=T\left(\mathbf{x}-\mathbf{x}^{(k-1)}\right),
% 		$$
% 		por lo que
% 		$$
% 			\mathbf{x}-\mathbf{x}^{(k)}=T\left(\mathbf{x}-\mathbf{x}^{(k-1)}\right)=T^2\left(\mathbf{x}-\mathbf{x}^{(k-2)}\right)=\cdots=T^k\left(\mathbf{x}-\mathbf{x}^{(0)}\right)=T^k \mathbf{z}
% 		$$
% 		Por lo tanto, $\operatorname{lím}_{k \rightarrow \infty} T^k \mathbf{z}=\operatorname{lím}_{k \rightarrow \infty} T^k\left(\mathbf{x}-\mathbf{x}^{(0)}\right)=\operatorname{lím}_{k \rightarrow \infty}\left(\mathbf{x}-\mathbf{x}^{(k)}\right)=\mathbf{0}$. Pero $\mathbf{z} \in \mathbb{R}^n$ era arbitrario, por lo que mediante el teorema 7.17, $T$ es convergente y $\rho(T)<1$

% 		\begin{enumerate}
% 			\item[$\left(\Leftarrow\right)$]

% 				Sea $z$ un vector arbitrario y $x$ la única solución para $x=Tx+c$.
% 				Defina $x^{0}=x-z$, y, para $k\geq1$, $x^{k}=Tx^{\left(k-1\right)}+c$
% 		\end{enumerate}
% 	\end{proof}
% \end{frame}

\begin{frame}
	\frametitle{Formulación matemática del método de Sobrerrelajación sucesiva (SOR)}
	Sean $A\in\mathbb{C}^{n\times n}$ y $b$ un vector columna dado en
	$\mathbb{C}^{n}$.
	Busquemos la solución del sistema de ecuaciones lineales
	\begin{math}
		Ax=b.
	\end{math}

	El vector solución $x$ existe en $\mathbb{C}^{n}$ y es único sii
	$A$ es no singular, y viene dado por
	\begin{math}
		x=A^{-1}b.
	\end{math}

	Supongamos que $A$ es no singular, y además que sus entradas
	diagonales $a_{ii}$ son todos números complejos distintos de cero.

	Podemos expresar la matriz $A$ en sus partes diagonal o fuera de la
	diagonal, permita que $D$ sea la \alert{matriz diagonal} cuyas
	entradas diagonales sean las de $A$, $-L$ es la parte
	\alert{estrictamente triangular inferior} de $A$ y $-U$ es la parte
	\alert{estrictamente triangular superior} de $A$.
	\begin{equation*}
		\underbrace{
			\begin{bNiceMatrix}
				a_{11}  & a_{12}  & \cdots & a_{1 n} \\
				a_{21}  & a_{22}  & \cdots & a_{2 n} \\
				\vdots  & \vdots  &        & \vdots  \\
				a_{n 1} & a_{n 2} & \cdots & a_{n n}
			\end{bNiceMatrix}
		}_{A}=
		\underbrace{
			\begin{bNiceMatrix}
				a_{11} & 0      & \Cdots & 0      \\
				0      & a_{22} & \Ddots & \Vdots \\
				\Vdots & \Ddots & \Ddots & 0      \\
				0      & \Cdots & 0      & a_{nn}
			\end{bNiceMatrix}
		}_{D}
		-
		\underbrace{
			\begin{bNiceMatrix}
				0       & \Cdots &            & 0      \\
				-a_{21} & \Ddots &            & \Vdots \\
				\Vdots  & \Ddots &            &        \\
				-a_{n1} & \Cdots & -a_{n,n-1} & 0
			\end{bNiceMatrix}
		}_{L}
		-
		\underbrace{
			\begin{bNiceMatrix}
				0      & -a_{12} & \Cdots & a_{1n}     \\
				\Vdots & \Ddots  & \Ddots & \Vdots     \\
				       &         &        & -a_{n-1,n} \\
				0      & \Cdots  &        & 0
			\end{bNiceMatrix}
		}_{U}.
	\end{equation*}
	% donde $D=\operatorname{diag}\left\{a_{11},\ldots,a_{nn}\right\}$, y
	% Para derivar los métodos iterativos clásicos ,asumiendo que el $\det\left(D\right)\neq0$,
	% $L$ y $U$ son matrices $n\times n$, respectivamente, estrictamente triangular
	% inferior y estrictamente triangular superior, cuyas entradas son las entradas negativas de $A$,
	% respectivamente, arriba y abajo de la diagonal principal de $A$.
	Con esta notación y un parámetro
	\begin{math}
		\omega\in\mathbb{C}
		\setminus\left\{0\right\}
	\end{math}
	llamado \alert{factor de relajación}, $A$ puede reescribirse como
	% el sistema de ecuaciones lineales $Ax=b$
	% , o $\left(D-L-U\right)x=b$,
	% \begin{equation*}
	% 	Dx=\left(L+U\right)x+b.
	% \end{equation*}
	\begin{align*}
		A  & =
		D-L-U=
		\left(\frac{\alert{1}+\alert{\omega-1}}{\alert{\omega}}\right)D-L-U=
		\boxed{
			\left(\frac{\alert{1}}{\alert{\omega}}D - L\right)+
			\left(\frac{\alert{\omega-1}}{\alert{\omega}}D-U\right)
		},
		\shortintertext{y así}
		Ax & =
		\left[
			\left(\frac{\alert{1}}{\alert{\omega}}D - L\right)+
			\left(\frac{\alert{\omega-1}}{\alert{\omega}}D-U\right)
			\right]x=
		\left(\frac{\alert{1}}{\alert{\omega}}D-L\right)x+
		\left(\frac{\alert{\omega-1}}{\alert{\omega}}D-U\right)x
		= b.
	\end{align*}
\end{frame}

\begin{frame}
	En consecuencia, el sistema de ecuaciones lineales $Ax=b$ se convierte en
	\begin{align*}
		\left(\frac{\alert{1}}{\alert{\omega}}D-L\right)x
		 & = -\left(\frac{\alert{\omega-1}}{\alert{\omega}}D-U\right)x + b. \\
		x
		 & ={\left(\frac{\alert{1}}{\alert{\omega}}D-L\right)}^{-1}
		\left[
			-\left(\frac{\alert{\omega-1}}{\alert{\omega}}D-U\right)x+
			b
		\right].                                                            \\
		x
		 & =
		-{\left(\frac{\alert{1}}{\alert{\omega}}D-L\right)}^{-1}
		\left(\frac{\alert{\omega-1}}{\alert{\omega}}D-U\right)x+
		{\left(\frac{\alert{1}}{\alert{\omega}}D-L\right)}^{-1}
		b.                                                                  \\
		% x
		%  & =
		% -{\left(\frac{\alert{1}}{\alert{\omega}}D - L\right)}^{-1}
		% \left(\frac{\alert{\omega-1}}{\alert{\omega}}D-U\right)x+
		% {\left(\frac{\alert{1}}{\alert{\omega}}D - L\right)}^{-1}
		% b.                                                                  \\
		x
		 & =
		-{\left(\frac{D-\alert{\omega}L}{\alert{\omega}}\right)}^{-1}
		\left(\frac{\left(\alert{\omega-1}\right)D-\alert{\omega}U}{\alert{\omega}}\right)x+
		{\left(\frac{D-\alert{\omega}L}{\alert{\omega}}\right)}^{-1}
		b.                                                                  \\
		x
		 & =
		-{\left(\frac{1}{\alert{\omega}}\right)}^{-1}{\left(D-\alert{\omega}L\right)}^{-1}
		\left(\frac{1}{\alert{\omega}}\right)\left(\left(\alert{\omega-1}\right)D-\alert{\omega}U\right)x+
		{\left(\frac{1}{\alert{\omega}}\right)}^{-1}{\left(D-\alert{\omega}L\right)}^{-1}
		b.                                                                  \\
		\Aboxed{
		x
		 & =
		\underbrace{
			{\left(D-\alert{\omega}L\right)}^{-1}
			\left(\left(\alert{1-\omega}\right)D+\alert{\omega}U\right)
		}_{T_{\omega}}x+
		\underbrace{
			\alert{\omega}{\left(D-\alert{\omega}L\right)}^{-1}b
		}_{c_{\omega}}.
		}
	\end{align*}

	En notación de un esquema iterativo resulta el \alert{método de SOR}
	\begin{equation*}
		\boxed{
		\forall k\geq0\colon
		x^{\left(k+1\right)}=
		T_{\omega}x^{\left(k\right)}+
		c_{\omega}.
		}\quad\text{o}\quad
		\boxed{
		\forall k\geq0\colon
		x_{i}^{\left(k+1\right)}=
		\frac{w}{a_{ii}}
		\left(
		b_{i}-
		\sum_{j=1}^{i-1}a_{ij}x_j^{\left(k+1\right)}-
		\sum_{j=i+1}^{n}a_{ij}x_{j}^{\left(k\right)}\right)+
		\left(1-w\right)x_{i}^{\left(k\right)}.
		}
	\end{equation*}
\end{frame}

\begin{frame}

	Para cada método iterativo, podemos asociar el vector error
	\begin{math}
		\epsilon^{\left(k\right)}\coloneqq
		x^{\left(k\right)}-x
	\end{math},
	donde $x$ es el único vector solución y podemos expresar los
	vectores error $\epsilon^{\left(k\right)}$ como
	\begin{equation*}
		\forall k\geq0:
		\epsilon^{\left(k\right)}=
		T\epsilon^{\left(k-1\right)}=
		\cdots=
		T^{\left(k\right)}\epsilon^{\left(0\right)}.
	\end{equation*}
	y por el \alert{criterio de convergencia}, los vectores error
	$\epsilon^{\left(k\right)}$ tienden al vector cero para todas las
	elecciones de $\epsilon^{\left(0\right)}$ sii el radio espectral
	$\rho\left(T\right)$ de la matriz del método es menor que uno.

	\begin{theorem}[Kahan]
		Una condición necesaria para que el \alert{método de SOR} converja
		es $\left|\omega-1\right|<1$.
		(Para $\omega\in\mathbb{R}$ esta condición resulta
		$\omega\in\left(0,2\right)$.)
	\end{theorem}

	\begin{definition}
		Una matriz $A\in\mathbb{C}^{n\times n}$ se dice que es Hermitiana
		sii $A^{\ast}=A$, donde el superíndice $\ast$ denota la
		transpuesta conjugada compleja.
		(Una matriz Hermitiana real es una matriz simétrica real y que
		cumple $A^{T}=A$, donde $T$ denota la transpuesta)
	\end{definition}

	\begin{definition}
		Una matriz Hermitiana $A\in\mathbb{C}^{n\times n}$ se dice que
		es definida positiva sii $\forall x^{\ast}Ax>0$,
		$\forall x\in\mathbb{C}^{n}\setminus\left\{0\right\}$.
		(Para $A$ real simétrica, la condición resulta
		$\forall x^{T}Ax>0$,
		$\forall x\in\mathbb{R}^{n}\setminus\left\{0\right\}$)
	\end{definition}

	\begin{theorem}[Reich - Ostrowski - Varga]
		Sea $A=D-E-E^{\ast}\in\mathbb{C}^{n\times n}$ Hermitiana,
		$D$ es Hermitiana y definida positiva, y
		$\det\left(D-\omega E\right)\neq0$,
		$\forall\omega\left(0,2\right)$.
		Entonces, $\rho\left(T_{\omega}\right)<1$ sii $A$ es definida
		positiva y $\omega\in\left(0,2\right)$.
	\end{theorem}
\end{frame}

\begin{frame}
	\begin{enumerate}\setcounter{enumi}{10}
		\item

		      Resuelva el siguiente sistema lineal
		      \begin{math}
			      \systeme{
			      4x_{1}-
			      x_{2}-
			      x_{4}=
			      0,
			      -x_{1}+
			      4x_{2}-
			      x_{3}-
			      x_{5}=
			      5,
			      -x_{2}+
			      4x_{3}-
			      x_{6}=
			      0,
			      -x_{1}+
			      4x_{4}-
			      x_{5}=
			      6,
			      -x_{2}-
			      x_{4}+
			      4x_{5}-
			      x_{6}=
			      -2,
			      -x_{3}-
			      x_{5}+
			      4x_{6}=
			      6
			      }
		      \end{math}
		      tiene solución
		      \begin{math}
			      \begin{pNiceMatrix}
				      1 \\
				      2 \\
				      1 \\
				      2 \\
				      1 \\
				      2
			      \end{pNiceMatrix}
		      \end{math}.

		      Resuelva el sistema lineal mediante los
		      \alert{métodos de SOR} y del \alert{descenso más rápido}
		      con una aritmética de redondeo a tres dígitos.
	\end{enumerate}
	\begin{solution}
		Sea $Ax=b$ el sistema lineal, donde
		\begin{equation*}
			A=
			\begin{bNiceMatrix}
				4  & -1 & 0  & -1 & 0  & 0  \\
				-1 & 4  & -1 & 0  & -1 & 0  \\
				0  & -1 & 4  & 0  & 0  & -1 \\
				-1 & 0  & 0  & 4  & -1 & 0  \\
				0  & -1 & 0  & -1 & 4  & -1 \\
				0  & 0  & -1 & 0  & -1 & 4
			\end{bNiceMatrix},\quad
			x=
			\begin{bNiceMatrix}
				x_{1} \\
				x_{2} \\
				x_{3} \\
				x_{4} \\
				x_{5} \\
				x_{6}
			\end{bNiceMatrix}\text{ y }
			b=
			\begin{bNiceMatrix}
				0  \\
				5  \\
				0  \\
				6  \\
				-2 \\
				6
			\end{bNiceMatrix}.
		\end{equation*}
	\end{solution}
\end{frame}
% highlightlines={18-19,23,25},
\begin{frame}
	\begin{solution}
		\begin{minipage}{0.45\textwidth}
			Por el \alert{método de SOR}.
			\begin{listing}[H]
				\inputminted[
					fontsize=\scriptsize,
					breaklines,
					firstline=1,
					lastline=35
				]{text}{resultado_pregunta11.txt}
			\end{listing}
		\end{minipage}
		\begin{minipage}{0.45\textwidth}
			\begin{listing}[H]
				\inputminted[
					fontsize=\scriptsize,
					breaklines,
					firstline=13,
					lastline=16
				]{python}{pregunta11.py}
				\inputminted[
					fontsize=\scriptsize,
					breaklines,
					firstline=19,
					lastline=25
				]{python}{pregunta11.py}
				\inputminted[
					fontsize=\scriptsize,
					breaklines,
					firstline=28,
					lastline=30
				]{python}{pregunta11.py}
				\inputminted[
					fontsize=\scriptsize,
					breaklines,
					firstline=33,
					lastline=35
				]{python}{pregunta11.py}
				\inputminted[
					fontsize=\scriptsize,
					breaklines,
					firstline=38,
					lastline=40
				]{python}{pregunta11.py}
				\inputminted[
					fontsize=\scriptsize,
					breaklines,
					firstline=43,
					lastline=57
				]{python}{pregunta11.py}
			\end{listing}
		\end{minipage}
	\end{solution}
\end{frame}

\begin{frame}
	\begin{solution}
		\begin{minipage}{0.45\textwidth}
			Por el \alert{método de SOR}.
			\begin{listing}[H]
				\inputminted[
					fontsize=\scriptsize,
					breaklines,
					firstline=36,
					lastline=37
				]{text}{resultado_pregunta11.txt}
			\end{listing}
		\end{minipage}
		\begin{minipage}{0.45\textwidth}
			\begin{listing}[H]
				\inputminted[
					fontsize=\scriptsize,
					escapeinside=||,
					mathescape=true,
					breaklines,
					firstline=68,
					lastline=74
				]{python}{pregunta11.py}
				\inputminted[
					fontsize=\scriptsize,
					escapeinside=||,
					mathescape=true,
					breaklines,
					firstline=77,
					lastline=84
				]{python}{pregunta11.py}
				\inputminted[
					fontsize=\scriptsize,
					escapeinside=||,
					mathescape=true,
					breaklines,
					firstline=87,
					lastline=102
				]{python}{pregunta11.py}
			\end{listing}
		\end{minipage}
	\end{solution}
\end{frame}

\begin{frame}
	\begin{solution}
		\begin{minipage}{0.45\textwidth}
			Por el \alert{método del descenso más rápido}.
			% \begin{listing}[H]
			% 	\inputminted[
			% 		fontsize=\scriptsize,
			% 		breaklines,
			% 		firstline=36,
			% 		lastline=37
			% 	]{text}{resultado_descensorapido.txt}
			% \end{listing}
		\end{minipage}
		\begin{minipage}{0.45\textwidth}
			\begin{listing}[H]
				\inputminted[
					fontsize=\tiny,
					breaklines,
					firstline=16,
					lastline=50
				]{python}{descensorapido.py}
			\end{listing}
		\end{minipage}
	\end{solution}
\end{frame}

\begin{frame}
	\begin{enumerate}\setcounter{enumi}{11}
		\item

		      Use los métodos de SOR y del descenso más rápido para
		      encontrar la solución de sistema $Ax=b$ con una precisión
		      de $10^{-5}$ en la norma ${\left\|\cdot\right\|}_{\infty}$.
		      % \setcounter{MaxMatrixCols}{30}
		      \begin{equation*}
			      a_{ij}=
			      \begin{cases}
				      4,  & \text {si }j=1\text { e }i=1,\ldots,16,      \\
				      -1, & \text {si }
				      \begin{cases}
					      j=i+1 \text { e } i=1,2,3,5,6,7,9,10,11,13,14,15,  \\
					      j=i-1 \text { e } i=2,3,4,6,7,8,10,11,12,14,15,16, \\
					      j=i+4 \text { e } i=1,\ldots, 12,                  \\
					      j=i-4 \text { e } i=5,\ldots, 16,
				      \end{cases} \\
				      0,  & \text {en otro caso}.
			      \end{cases}\quad
			      \text{y}\quad
			      b=
			      \tiny
			      \begin{bNiceMatrix}
				      1.902207  \\
				      1.051143  \\
				      1.175689  \\
				      3.480083  \\
				      0.819600  \\
				      -0.264419 \\
				      -0.412789 \\
				      1.175689  \\
				      0.913337  \\
				      -0.150209 \\
				      -0.264419 \\
				      1.051143  \\
				      1.966694  \\
				      0.913337  \\
				      0.819600  \\
				      1.902207
			      \end{bNiceMatrix}
		      \end{equation*}
	\end{enumerate}
	\begin{solution}
		.
	\end{solution}
\end{frame}

\begin{frame}
	\begin{solution}
		\begin{minipage}{0.45\textwidth}
			Por el \alert{método del descenso más rápido}.
			% \begin{listing}[H]
			% 	\inputminted[
			% 		fontsize=\scriptsize,
			% 		breaklines,
			% 		firstline=36,
			% 		lastline=37
			% 	]{text}{resultado_descensorapido.txt}
			% \end{listing}
		\end{minipage}
		\begin{minipage}{0.45\textwidth}
			\begin{listing}[H]
				\inputminted[
					fontsize=\tiny,
					breaklines,
					firstline=16,
					lastline=50
				]{python}{descensorapido.py}
			\end{listing}
		\end{minipage}
	\end{solution}
\end{frame}

\begin{frame}
	\begin{enumerate}\setcounter{enumi}{12}
		\item


		      Use los métodos de SOR y del descenso más rápido para
		      encontrar la solución de sistema $Ax=b$ con una precisión
		      de $10^{-5}$ en la norma ${\left\|\cdot\right\|}_{\infty}$.

		      \begin{equation*}
			      a_{ij}=
			      \begin{cases}
				      4,  & \text {si }j=1\text { e }i=1,\ldots,25,                             \\
				      -1, & \text {si }
				      \begin{cases}
					      j=i+1 \text { e } i=1,2,3,4,6,7,8,9,11,12,13,14,16,17,18,19,21,22,23,24,  \\
					      j=i-1 \text { e } i=2,3,4,5,7,8,9,10,12,13,14,15,17,18,19,20,22,23,24,25, \\
					      j=i+5 \text { e } i=1,\ldots, 20,                                         \\
					      j=i-5 \text { e } i=6,\ldots, 25,
				      \end{cases} \\
				      0,  & \text {en otro caso}.
			      \end{cases}\quad
			      \text{y}\quad
			      b=
			      \tiny
			      \begin{bNiceMatrix}
				      1  \\
				      0  \\
				      -1 \\
				      0  \\
				      2  \\
				      1  \\
				      0  \\
				      -1 \\
				      0  \\
				      2  \\
				      1  \\
				      0  \\
				      -1 \\
				      0  \\
				      2  \\
				      1  \\
				      0  \\
				      -1 \\
				      0  \\
				      2  \\
				      1  \\
				      0  \\
				      -1 \\
				      0  \\
				      2  \\
			      \end{bNiceMatrix}
		      \end{equation*}
	\end{enumerate}
	\begin{solution}
		.
	\end{solution}
\end{frame}

\begin{frame}
	\begin{solution}
		\begin{minipage}{0.45\textwidth}
			\begin{listing}[H]
				\inputminted[
					fontsize=\scriptsize,
					breaklines,
					firstline=1,
					lastline=32
				]{text}{resultado_pregunta12.txt}
			\end{listing}
		\end{minipage}
		\begin{minipage}{0.45\textwidth}
			\begin{listing}[H]
				\inputminted[
					fontsize=\scriptsize,
					breaklines,
					firstline=4,
					lastline=20
				]{python}{pregunta12.py}
				\inputminted[
					fontsize=\scriptsize,
					breaklines,
					firstline=23,
					lastline=44
				]{python}{pregunta12.py}
			\end{listing}
		\end{minipage}
	\end{solution}
\end{frame}

\begin{frame}
	\begin{enumerate}\setcounter{enumi}{17}
		\item

		      Un fabricante de autos ha lanzado al mercado tres nuevos
		      modelos $A, B$ y $C$.
		      El precio de venta de cada modelo es de $15000$, $20000$ y
		      $30000$ soles, respectivamente, ascendiendo el importe
		      total de los autos vendidos durante el primer mes a
		      $250000$.
		      Donde los costes de fabricación son de S/$10000$ para el
		      modelo $A$, S/$15000$ para el modelo $B$ y S/$20000$ para
		      el modelo $C$, siendo el coste total de fabricación de los
		      autos en el mes de S/$175000$ y el número total de autos
		      vendidos es de $140$.
		      Determine el número de autos vendidos de cada modelo en el
		      mes, usando el programa desarrollado del gradiente conjugado.

	\end{enumerate}
	\begin{solution}

		\

		\begin{minipage}{0.55\textwidth}
			\begin{itemize}
				\item

				      $x$: representa el número de autos vendidos del modelo $A$.

				\item

				      $y$: representa el número de autos vendidos del modelo $B$.

				\item

				      $z$: representa el número de autos vendidos del modelo $C$.
			\end{itemize}
		\end{minipage}
		\begin{minipage}{0.35\textwidth}
			\systeme{
				15000x+20000y+30000z=250000,
				10000x+15000y+20000z=175000,
				x+y+z=140
			}
		\end{minipage}
	\end{solution}
\end{frame}

\begin{frame}
	\begin{solution}
		\begin{minipage}{0.45\textwidth}
			\begin{listing}[H]
				\inputminted[
					fontsize=\tiny,
					breaklines,
					firstline=1,
					lastline=10
				]{text}{resultado_pregunta18.txt}
			\end{listing}
		\end{minipage}
		\begin{minipage}{0.45\textwidth}
			\begin{listing}[H]
				\inputminted[
					fontsize=\scriptsize,
					breaklines,
					firstline=1,
					lastline=30
				]{python}{edwin_gradiente_conjugado.py}
			\end{listing}
		\end{minipage}
	\end{solution}
\end{frame}

\begin{frame}\transblindsvertical
	\frametitle{Referencias}
	%------------------------------------------------------------ 1
	\only<1>{
		\begin{itemize}
			\item Libros
			      \nocite{*}
			      \printbibliography[heading=none,keyword=book]
		\end{itemize}
	}
	%------------------------------------------------------------ 2
	\only<2>{
		\begin{itemize}
			\item Artículos científicos
			      \printbibliography[heading=none,keyword=paper]
		\end{itemize}
	}
	%------------------------------------------------------------ 3
	\only<3>{
		\begin{itemize}
			\item Sitios web
			      \printbibliography[heading=none,keyword=online]
		\end{itemize}
	}
\end{frame}

\begin{frame}
	The unique vector solution of this matrix problem is obviously
	$$
		x_1=x_2=x_3=x_4=1 .
	$$
	If our initial vector approximation is $\mathbf{x}^{(0)}=\mathbf{0}$, then the initial vector $\boldsymbol{\epsilon}^{(0)}$ has all its components equal to $-1$. For this case, the point Jacobi matrix and the point successive overrelaxation iterative matrix are given by
	$$
		B=\frac{1}{4}\left[\begin{array}{llll}
				0 & 0 & 1 & 1 \\
				0 & 0 & 1 & 1 \\
				1 & 1 & 0 & 0 \\
				1 & 1 & 0 & 0
			\end{array}\right], \quad \mathcal{L}_\omega=\left[\begin{array}{cccc}
				\sigma      & 0           & \tau            & \tau            \\
				0           & \sigma      & \tau            & \tau            \\
				\sigma \tau & \sigma \tau & 2 \tau^2+\sigma & 2 \tau^2        \\
				\sigma \tau & \sigma \tau & 2 \tau^2        & 2 \tau^2+\sigma
			\end{array}\right]
	$$
	where $\tau:=\omega / 4$, and $\sigma:=1-\omega$. If $\alpha^{(m)}$ denotes the error vectors for the point Jacobi iterative method, and $\boldsymbol{\beta}^{(m)}$ denotes the error vectors for the point Gauss-Seidel iterative method (the special case $\omega=1$ of the point successive overrelaxation iterative method), then it can also be verified that
	(3.24) $\boldsymbol{\alpha}^{(m)}=B^m \epsilon^{(0)}=\frac{-1}{2^m}\left[\begin{array}{l}1 \\ 1 \\ 1 \\ 1\end{array}\right], \quad \beta^{(m)}=\mathcal{L}_1^m \epsilon^{(0)}=\frac{-1}{4^m}\left[\begin{array}{l}2 \\ 2 \\ 1 \\ 1\end{array}\right], \quad m \geq 1$.
	Thus, as
	$$
		\left\|\boldsymbol{\alpha}^{(m)}\right\|=\frac{1}{2^{m-1}}>\left\|\boldsymbol{\beta}^{(m)}\right\|=\frac{\sqrt{10}}{4^m} \quad \text { for all } m \geq 1,
	$$
	we conclude that the vector norms $\left\|\beta^{(m)}\right\|$ tend more rapidly to zero than do the vector norms $\left\|\alpha^{(m)}\right\|$ for this particular choice of $\epsilon^{(0)}$. This in turn leads us to suspect that the point Gauss-Seidel iterative method is in some cases "faster" than the point Jacobi iterative method. The next section is
\end{frame}

\end{document}