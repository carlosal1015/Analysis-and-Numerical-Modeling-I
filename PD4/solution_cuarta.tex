%! Aldo Luna, Alexandra Mazzetti, Carlos Aznarán, Edward Canales
%! Universidad Nacional de Ingeniería
%! Facultad de Ciencias
%! Lima, Perú
%! Uso:
%! $ sudo pacman -Syu texlive-most zathura # dependencias, visor
%! $ arara solution_cuarta
%! $ zathura solution_cuarta.pdf
%! Ver https://wiki.archlinux.org/title/TeX_Live
% arara: clean: {
% arara: --> extensions:
% arara: --> ['aux','log','nav',
% arara: --> 'out','snm','toc','pytxcode','pdf']
% arara: --> }
% arara: lualatex: {
% arara: --> shell: yes,
% arara: --> draft: yes,
% arara: --> interaction: batchmode
% arara: --> }
% arara: biber
%! arara: pythontex
% arara: lualatex: {
% arara: --> shell: yes,
% arara: --> draft: yes,
% arara: --> interaction: batchmode
% arara: --> }
% arara: lualatex: {
% arara: --> shell: yes,
% arara: --> synctex: yes,
% arara: --> interaction: batchmode
% arara: --> }
% arara: clean: {
% arara: --> extensions:
% arara: --> ['aux','log','nav',
% arara: --> 'out','snm','toc','pytxcode']
% arara: --> }
\PassOptionsToPackage{svgnames}{xcolor}
\documentclass[
	spanish,
	8pt,
	utf8,
	xcolor=table,
	handout,
	aspectratio=169,
	professionalfonts,
	% notheorems,
	mathserif,
	leqno,
	% t
]{beamer}
\setbeamersize{text margin left=5pt,text margin right=5pt}
\usepackage[spanish,es-sloppy]{babel}
\spanishdatedel\decimalpoint
\usepackage{mathtools}
\usepackage{nicematrix}
\usepackage{systeme}
\usepackage{minted}
\usepackage{enumerate}
\usepackage{multicol}
% \usepackage{pythontex}
\usepackage[
	citestyle=numeric,
	style=apa,
	backend=biber,
	defernumbers=true,
	sorting=ynt,
	maxcitenames=4
]{biblatex}
\addbibresource{references.bib}

\newcolumntype{x}[1]{>{\centering\arraybackslash\hspace{0pt}}p{#1}}

\newcounter{savedenum}
\newcommand*{\saveenum}{\setcounter{savedenum}{\theenumi}}
\newcommand*{\resume}{\setcounter{enumi}{\thesavedenum}}

\setbeamertemplate{navigation symbols}{}
\setbeamertemplate{footline}{}
\setbeamertemplate{headline}{}

% https://tex.stackexchange.com/questions/68080/beamer-bibliography-icon
\setbeamertemplate{bibliography item}{%
	\ifboolexpr{ test {\ifentrytype{book}} or test {\ifentrytype{mvbook}}
		or test {\ifentrytype{collection}} or test {\ifentrytype{mvcollection}}
		or test {\ifentrytype{reference}} or test {\ifentrytype{mvreference}} }
	{\setbeamertemplate{bibliography item}[book]}
	{\ifentrytype{online}
		{\setbeamertemplate{bibliography item}[online]}
		{\setbeamertemplate{bibliography item}[article]}}%
	\usebeamertemplate{bibliography item}}

\defbibenvironment{bibliography}
{\list{}
	{\settowidth{\labelwidth}{\usebeamertemplate{bibliography item}}%
		\setlength{\leftmargin}{\labelwidth}%
		\setlength{\labelsep}{\biblabelsep}%
		\addtolength{\leftmargin}{\labelsep}%
		\setlength{\itemsep}{\bibitemsep}%
		\setlength{\parsep}{\bibparsep}}}
{\endlist}
{\item}

\makeatletter
\newenvironment<>{proofs}[1][\proofname]{%
    \par
    \def\insertproofname{#1\@addpunct{.}}%
    \usebeamertemplate{proof begin}#2}
  {\usebeamertemplate{proof end}}
\makeatother


\title{
	\huge\sffamily
	Cuarta Práctica Dirigida\quad Grupo N$^{\circ}$~3
}

\subtitle{
	\large\scshape
	Análisis y Modelamiento Numérico I\quad CM4F1 B\\[.5\baselineskip]
		\normalsize\normalfont
		Profesor Jonathan Munguia La Cotera
}

\author{
	Aldo Luna Bueno\quad\and\quad
	Alexandra Gutierrez Mazzetti\quad\and\quad
	Carlos Aznarán Laos\quad\and\quad
	Edward Canales Yarin
}

\institute{\large
	Facultad de Ciencias \and
	Universidad Nacional de Ingeniería
}
\date{16 de noviembre del 2022}

\begin{document}

\frame{\titlepage}

\begin{frame}
	Para estudiar la convergencia de técnicas de iteración general, necesitamos analizar la fórmula
	\begin{equation*}
		\boxed{
			\forall k\in\mathbb{N}:
			x^{\left(k\right)}=
			Tx^{\left(k-1\right)}+c.
		}
	\end{equation*}

	\begin{theorem}[]
		Para cualquier $x^{\left(0\right)}\in\mathbb{R}^{n}$, la sucesión
		\begin{math}
			{\left\{x^{\left(k\right)}\right\}}_{k\in\mathbb{N}}
		\end{math}
		definida por
		\begin{equation*}
			x^{\left(k\right)}=
			Tx^{\left(k-1\right)}+
			c.
		\end{equation*}
		converge a la \alert{solución única} de $x=Tx+c$ sii
		$\rho\left(T\right)<1$.

		\begin{proof}
			\begin{itemize}
				\item[$\left(\Rightarrow\right)$]

					Primero suponga que $\rho\left(T\right)<1$.
					Entonces,
					\begin{align*}
						x^{\left(k\right)}                       & =
						T\alert{x^{\left(k-1\right)}}+c                                                                       \\
						                                         & =
						T\left(T x^{\left(k-2\right)}+c\right)+c                                                              \\
						                                         & =T^{2}x^{\left(k-2\right)}+\left(T+I\right)c               \\
						                                         & \vdotswithin{=}                                            \\
						                                         & =T^{k}x^{\left(0\right)}+\left(T^{k-1}+\cdots+T+I\right)c.
						\shortintertext{Puesto que $\rho\left(T\right)<1$, implica que $T$ es convergente y}
						\lim_{k\to\infty}T^{k}x^{\left(0\right)} & =0.
					\end{align*}
			\end{itemize}
		\end{proof}
	\end{theorem}
\end{frame}

\begin{frame}
	\begin{proof}
		\begin{equation*}
			\lim_{k\to\infty}
			x^{\left(k\right)}=
			\lim_{k\to\infty}
			T^{k}x^{\left(0\right)}+
			\left(\sum_{j=0}^{\infty}T^{j}\right)c=
			0+{\left(I-T\right)}^{-1}c=
				{\left(I-T\right)}^{-1}c
		\end{equation*}
		Por lo tanto, la sucesión $\left\{\mathbf{x}^{(k)}\right\}$ converge al vector $\mathbf{x} \equiv(I-T)^{-1} \mathbf{c}$ y $\mathbf{x}=T \mathbf{x}+\mathbf{c}$.
		Para probar lo contrario, mostraremos que para cualquier $\mathbf{z} \in \mathbb{R}^n$, tenemos $\operatorname{lím}_{k \rightarrow \infty}$ $T^k \mathbf{z}=\mathbf{0}$. Con el teorema 7.17, esto es equivalente a $\rho(T)<1$.

		Sea $\mathbf{z}$ un vector arbitrario y $x$ la única solución para $\mathbf{x}=T \mathbf{x}+\mathbf{c}$. Defina $\mathbf{x}^{(0)}=\mathbf{x}-\mathbf{z}$, y, para $k \geq 1, \mathbf{x}^{(k)}=T \mathbf{x}^{(k-1)}+\mathbf{c}$. Entonces $\left\{\mathbf{x}^{(k)}\right\}$ converge a $\mathbf{x}$. También,
		$$
			\mathbf{x}-\mathbf{x}^{(k)}=(T \mathbf{x}+\mathbf{c})-\left(T \mathbf{x}^{(k-1)}+\mathbf{c}\right)=T\left(\mathbf{x}-\mathbf{x}^{(k-1)}\right),
		$$
		por lo que
		$$
			\mathbf{x}-\mathbf{x}^{(k)}=T\left(\mathbf{x}-\mathbf{x}^{(k-1)}\right)=T^2\left(\mathbf{x}-\mathbf{x}^{(k-2)}\right)=\cdots=T^k\left(\mathbf{x}-\mathbf{x}^{(0)}\right)=T^k \mathbf{z}
		$$
		Por lo tanto, $\operatorname{lím}_{k \rightarrow \infty} T^k \mathbf{z}=\operatorname{lím}_{k \rightarrow \infty} T^k\left(\mathbf{x}-\mathbf{x}^{(0)}\right)=\operatorname{lím}_{k \rightarrow \infty}\left(\mathbf{x}-\mathbf{x}^{(k)}\right)=\mathbf{0}$. Pero $\mathbf{z} \in \mathbb{R}^n$ era arbitrario, por lo que mediante el teorema 7.17, $T$ es convergente y $\rho(T)<1$

		\begin{enumerate}
			\item[$\left(\Leftarrow\right)$]

				Sea $z$ un vector arbitrario y $x$ la única solución para $x=Tx+c$.
				Defina $x^{0}=x-z$, y, para $k\geq1$, $x^{k}=Tx^{\left(k-1\right)}+c$
		\end{enumerate}
	\end{proof}
\end{frame}

\begin{frame}
	\frametitle{Método de Sobrerrelajación sucesiva (SOR)}
	El \alert{método SOR} se puede escribir de la forma
	\begin{math}
		x^{\left(k\right)}=
		T_{\omega}x^{\left(k-1\right)}+
		c_{\omega}
	\end{math}
	al dividir $A$ en sus partes diagonal o fuera de la diagonal.
	Para observar esto, permite que $D$ sea la \alert{matriz diagonal}
	cuyas entradas diagonales sean las de $A$, $-L$ es la parte
	\alert{estrictamente triangular inferior} de $A$ y $-U$ es la parte
	\alert{estrictamente triangular superior} de $A$.
	Con esta notación,
	\begin{equation*}
		\underbrace{
			\begin{bNiceMatrix}
				a_{11}  & a_{12}  & \cdots & a_{1 n} \\
				a_{21}  & a_{22}  & \cdots & a_{2 n} \\
				\vdots  & \vdots  &        & \vdots  \\
				a_{n 1} & a_{n 2} & \cdots & a_{n n}
			\end{bNiceMatrix}
		}_{A}=
		\underbrace{
			\begin{bNiceMatrix}
				a_{11} & 0      & \Cdots & 0      \\
				0      & a_{22} & \Ddots & \Vdots \\
				\Vdots & \Ddots & \Ddots & 0      \\
				0      & \Cdots & 0      & a_{nn}
			\end{bNiceMatrix}
		}_{D}
		-
		\underbrace{
			\begin{bNiceMatrix}
				0       & \Cdots &            & 0      \\
				-a_{21} & \Ddots &            & \Vdots \\
				\Vdots  & \Ddots &            &        \\
				-a_{n1} & \Cdots & -a_{n,n-1} & 0
			\end{bNiceMatrix}
		}_{L}
		-
		\underbrace{
			\begin{bNiceMatrix}
				0      & -a_{12} & \Cdots & a_{1n}     \\
				\Vdots & \Ddots  & \Ddots & \Vdots     \\
				       &         &        & -a_{n-1,n} \\
				0      & \Cdots  &        & 0
			\end{bNiceMatrix}
		}_{U}
	\end{equation*}
	Entonces, el sistema de ecuaciones lineales $Ax=b$, o $\left(D-L-U\right)x=b$, puede reescribirse como
	\begin{equation*}
		Dx=\left(L+U\right)x+b.
	\end{equation*}
	para un parámetro
	\begin{math}
		\omega\in\mathbb{C}
		\setminus\left\{0\right\}
	\end{math}
	llamado \alert{factor de relajación}.
	\begin{align*}
		A=\left(\frac{1}{w}D - L\right)+
		\left(\frac{w-1}{w}\right)
	\end{align*}
\end{frame}

\begin{frame}
	Para derivar los métodos iterativos clásicos uno escribe $A=D-L-U$, con
	$D=\operatorname{diag}\left(A\right)$, asumiendo que el $\det\left(D\right)\neq0$,
	$L$ y $U$ son matrices estrictamente inferior y estrictamente superior, respectivamente.

	\begin{theorem}[Kahan]
		Una condición necesaria para que el método SOR converja es $\left|\omega-1\right|<1$.
		(Para $\omega\in\mathbb{R}$ esta condición resulta $\omega\in\left(0,2\right)$.)
	\end{theorem}

	\begin{definition}
		Una matriz $A\in\mathbb{C}^{n\times n}$ se dice que es Hermitiana sii
		$A^{\ast}=A$, donde el superíndice $\ast$ denota la transpuesta conjugada compleja.
		(Una matriz Hermitiana real es una matriz simétrica real y que cumple $A^{T}=A$, donde $T$ denota la transpuesta)
	\end{definition}

	\begin{definition}
		Una matriz Hermitiana $A\in\mathbb{C}^{n\times n}$ se dice que
		es definida positiva sii $\forall x^{\ast}Ax>0$,
		$\forall x\in\mathbb{C}^{n}\setminus\left\{0\right\}$.
		(Para $A$ real simétrica, la condición resulta $\forall x^{T}Ax>0$,
		$\forall x\in\mathbb{R}^{n}\setminus\left\{0\right\}$)
	\end{definition}

	\begin{theorem}[Reich - Ostrowski - Varga]
		Sea $A=D-E-E^{\ast}\in\mathbb{C}^{n\times n}$ Hermitiana,
		$D$ es Hermitiana y definida positiva, y $\det\left(D-\omega E\right)\neq0$,
		$\forall\omega\left(0,2\right)$.
		Entonces, $\rho\left(T_{\omega}\right)<1$ sii $A$ es definida positiva y $\omega\in\left(0,2\right)$.
	\end{theorem}
\end{frame}

\begin{frame}

	\begin{enumerate}\setcounter{enumi}{10}
		\item

		      Resuelva el siguiente sistema lineal
		      \begin{math}
			      \systeme{
			      4x_{1}-
			      x_{2}-
			      x_{4}=
			      0,
			      -x_{1}+
			      4x_{2}-
			      x_{3}-
			      x_{5}=
			      5,
			      -x_{2}+
			      4x_{3}-
			      x_{6}=
			      0,
			      -x_{1}+
			      4x_{4}-
			      x_{5}=
			      6,
			      -x_{2}-
			      x_{4}+
			      4x_{5}-
			      x_{6}=
			      -2,
			      -x_{3}-
			      x_{5}+
			      4x_{6}=
			      6
			      }
		      \end{math}
		      tiene solución ${\left(1,-1,1\right)}^{T}$.

		      Resuelva el sistema lineal mediante los métodos de SOR y
		      del descenso más rápido con una aritmética de redondeo a
		      tres dígitos.

	\end{enumerate}

	\begin{solution}
		.
	\end{solution}
\end{frame}

\begin{frame}\transblindsvertical
	\frametitle{Referencias}
	%------------------------------------------------------------ 1
	\only<1>{
		\begin{itemize}
			\item Libros
			      \nocite{*}
			      \printbibliography[heading=none,keyword=book]
		\end{itemize}
	}
	%------------------------------------------------------------ 2
	\only<2>{
		\begin{itemize}
			\item Artículos científicos
			      \printbibliography[heading=none,keyword=paper]
		\end{itemize}
	}
	%------------------------------------------------------------ 3
	\only<3>{
		\begin{itemize}
			\item Sitios web
			      \printbibliography[heading=none,keyword=online]
		\end{itemize}
	}
\end{frame}

\end{document}