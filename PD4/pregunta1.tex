%! Carlos Aznarán
%! Universidad Nacional de Ingeniería
%! Facultad de Ciencias
%! Lima, Perú
%! Uso:
%! $ sudo pacman -Syu texlive-most zathura # dependencias, visor
%! $ arara pregunta2
%! $ zathura pregunta2.pdf
%! Ver https://wiki.archlinux.org/title/TeX_Live
% arara: clean: {
% arara: --> extensions:
% arara: --> ['aux','log','nav',
% arara: --> 'out','snm','toc','pytxcode','pdf']
% arara: --> }
% arara: lualatex: {
% arara: --> shell: yes,
% arara: --> draft: yes,
% arara: --> interaction: batchmode
% arara: --> }
%! arara: biber
%! arara: pythontex
% arara: lualatex: {
% arara: --> shell: yes,
% arara: --> draft: yes,
% arara: --> interaction: batchmode
% arara: --> }
% arara: lualatex: {
% arara: --> shell: yes,
% arara: --> synctex: yes,
% arara: --> interaction: batchmode
% arara: --> }
% arara: clean: {
% arara: --> extensions:
% arara: --> ['aux','log','nav',
% arara: --> 'out','snm','toc','pytxcode']
% arara: --> }
\PassOptionsToPackage{svgnames}{xcolor}
\documentclass[
	spanish,
	8pt,
	utf8,
	xcolor=table,
	handout,
	aspectratio=169,
	professionalfonts,
	% notheorems,
	mathserif,
	leqno,
	% t
]{beamer}
\setbeamersize{text margin left=5pt,text margin right=5pt}
\usepackage[spanish,es-sloppy]{babel}
\spanishdatedel\decimalpoint
\usepackage{mathtools}
\usepackage{nicematrix}
\usepackage{systeme}
\usepackage{minted}
\usepackage{enumerate}
\usepackage{multicol}
% \usepackage{pythontex}
\usepackage[
	citestyle=numeric,
	style=apa,
	backend=biber,
	defernumbers=true,
	sorting=ynt,
	maxcitenames=4
]{biblatex}
\addbibresource{references.bib}

\newcolumntype{x}[1]{>{\centering\arraybackslash\hspace{0pt}}p{#1}}

\newcounter{savedenum}
\newcommand*{\saveenum}{\setcounter{savedenum}{\theenumi}}
\newcommand*{\resume}{\setcounter{enumi}{\thesavedenum}}

\setbeamertemplate{navigation symbols}{}
\setbeamertemplate{footline}{}
\setbeamertemplate{headline}{}

% https://tex.stackexchange.com/questions/68080/beamer-bibliography-icon
\setbeamertemplate{bibliography item}{%
	\ifboolexpr{ test {\ifentrytype{book}} or test {\ifentrytype{mvbook}}
		or test {\ifentrytype{collection}} or test {\ifentrytype{mvcollection}}
		or test {\ifentrytype{reference}} or test {\ifentrytype{mvreference}} }
	{\setbeamertemplate{bibliography item}[book]}
	{\ifentrytype{online}
		{\setbeamertemplate{bibliography item}[online]}
		{\setbeamertemplate{bibliography item}[article]}}%
	\usebeamertemplate{bibliography item}}

\defbibenvironment{bibliography}
{\list{}
	{\settowidth{\labelwidth}{\usebeamertemplate{bibliography item}}%
		\setlength{\leftmargin}{\labelwidth}%
		\setlength{\labelsep}{\biblabelsep}%
		\addtolength{\leftmargin}{\labelsep}%
		\setlength{\itemsep}{\bibitemsep}%
		\setlength{\parsep}{\bibparsep}}}
{\endlist}
{\item}

\makeatletter
\newenvironment<>{proofs}[1][\proofname]{%
    \par
    \def\insertproofname{#1\@addpunct{.}}%
    \usebeamertemplate{proof begin}#2}
  {\usebeamertemplate{proof end}}
\makeatother


\title{
	\huge\sffamily
	Lista N$^{\circ}1$
}

\subtitle{
	\large\scshape
	Análisis y Modelamiento Numérico I\quad CM4F1 B\\[.5\baselineskip]
		\normalsize\normalfont
		Profesor Ángel Enrique Ramírez Gutiérrez.
}

\author{
	Carlos A. Aznarán Laos
}

\institute{\large
	Facultad de Ciencias \and
	Universidad Nacional de Ingeniería
}
\date{\today}

\begin{document}

\frame{\titlepage}

\begin{frame}
	\begin{enumerate}\setcounter{enumi}{0}

		\item

		      Para
		      \begin{math}
			      A=
			      \begin{bNiceMatrix}
				      3 & -2 \\
				      0 & 3  \\
				      4 & 4
			      \end{bNiceMatrix}
			      \in\mathbb{R}^{3\times 2}
		      \end{math}
		      y
		      \begin{math}
			      b=
			      \begin{bNiceMatrix}
				      3 \\
				      5 \\
				      4
			      \end{bNiceMatrix}
			      \in\mathbb{R}^{3}
		      \end{math}.
		      Encuentre el
		      \begin{math}
			      \operatornamewithlimits{argmin}_{x\in\mathbb{R}^{n}}
			      \left\|b-Ax\right\|
		      \end{math}.
	\end{enumerate}

	\begin{solution}
		Como $3>2$, y el  $\operatorname{rango}\left(A\right)=2$ porque
		es el número máximo de vectores linealmente independientes
		respecto al espacio columnas de $A$.
		Calculemos la factorización $QR$ de la matriz $A$ vía el proceso de Gram-Schmidt:

		Sean
		\begin{math}
			v_{1}=
			\begin{bNiceMatrix}
				3 \\
				0 \\
				4
			\end{bNiceMatrix}
		\end{math}
		y
		\begin{math}
			v_{2}=
			\begin{bNiceMatrix}
				-2 \\
				3  \\
				4
			\end{bNiceMatrix}
		\end{math}.
		Entonces,
		\begin{align*}
			u_{1} & =
			v_{1}=
			\begin{bNiceMatrix}
				3 \\
				0 \\
				4
			\end{bNiceMatrix}.
			      & \implies q_{1}=\frac{u_{1}}{\left\|u_{1}\right\|}=
			\begin{bNiceMatrix}
				0.6 \\
				0   \\
				0.8
			\end{bNiceMatrix}.                                         \\
			u_{2} & =
			v_{2}-
			\left\langle q_{1},v_{2}\right\rangle q_{1}=
			\begin{bNiceMatrix}
				-2 \\
				3  \\
				4
			\end{bNiceMatrix}-
			\left\langle
			\begin{bNiceMatrix}
				0.6 \\
				0   \\
				0.8
			\end{bNiceMatrix},
			\begin{bNiceMatrix}
				-2 \\
				3  \\
				4
			\end{bNiceMatrix}
			\right\rangle \begin{bNiceMatrix}
				              0.6 \\
				              0   \\
				              0.8
			              \end{bNiceMatrix}.
			\\
			u_{2}
			      & =
			\begin{bNiceMatrix}
				-2 \\
				3  \\
				4
			\end{bNiceMatrix}-
			2
			\begin{bNiceMatrix}
				0.6 \\
				0   \\
				0.8
			\end{bNiceMatrix}=
			\begin{bNiceMatrix}
				-3.2 \\
				3    \\
				2.4
			\end{bNiceMatrix}.
			      & \implies q_{2}=\frac{u_{2}}{\left\|u_{2}\right\|}=
			\begin{bNiceMatrix}
				-0.64 \\
				0.6   \\
				0.48
			\end{bNiceMatrix}.
		\end{align*}
	\end{solution}
\end{frame}

\begin{frame}
	\begin{solution}
		Entonces, por el método de factorización $QR$ obtenemos las
		matrices
		\begin{equation*}
			Q =
			\begin{bNiceMatrix}
				q_{1} & q_{2}
			\end{bNiceMatrix}=
			\begin{bNiceMatrix}
				0.6 & -0.64 \\
				0   & 0.6   \\
				0.8 & 0.48
			\end{bNiceMatrix}
			\text{ y }
			R =
			Q^{T}A=
			\begin{bNiceMatrix}
				0.6 & -0.64 \\
				0   & 0.6   \\
				0.8 & 0.48
			\end{bNiceMatrix}^{T}
			\begin{bNiceMatrix}
				3 & -2 \\
				0 & 3  \\
				4 & 4
			\end{bNiceMatrix}=
			\begin{bNiceMatrix}
				5 & 2 \\
				0 & 5
			\end{bNiceMatrix}.
		\end{equation*}
	\end{solution}
	Por otro lado, obtenemos estos productos:
	\begin{equation*}
		QA = R =
		\begin{bNiceMatrix}
			R_{1} \\
			0
		\end{bNiceMatrix}=
		\begin{bNiceMatrix}
			5 & 2 \\
			0 & 5
		\end{bNiceMatrix}=
		R_{1}
		\text{ y }
		Qb =
		\begin{bNiceMatrix}
			c \\
			d
		\end{bNiceMatrix}=
		\begin{bNiceMatrix}
			5 \\
			3
		\end{bNiceMatrix}=
		c.
	\end{equation*}
	Además, encontremos el mínimo valor de
	\begin{math}
		\left\|Ax-b\right\|=
		\left\|QAx-Qb\right\|=
		\left\|\begin{bNiceMatrix}
			R_{1}x-c \\
			d
		\end{bNiceMatrix}\right\|=
		\sqrt{
		{\left\|R_{1}x-c\right\|}^{2}+
		{\left\|d\right\|}^{2}}
	\end{math}, $\forall x\in\mathbb{R}^{n}$.

	Esto ocurre cuando ${\left\|R_{1}x-c\right\|}^{2}$ sea mínimo, por
	ello, basta resolver el sistema lineal $R_{1}x=c$.

	Finalmente, resolvemos este último sistema por sustitución regresiva:
	\begin{equation*}
		\begin{bNiceMatrix}
			5 & 2 \\
			0 & 5
		\end{bNiceMatrix}
		\begin{bNiceMatrix}
			x_{1} \\
			x_{2}
		\end{bNiceMatrix}=
		\begin{bNiceMatrix}
			5 \\
			3
		\end{bNiceMatrix}
	\end{equation*}
	obteniendo como resultado
	\begin{math}
		\overline{x}=
		\begin{bNiceMatrix}
			x_{1} \\
			x_{2}
		\end{bNiceMatrix}=
		\begin{bNiceMatrix}
			0.76 \\
			0.6
		\end{bNiceMatrix}
	\end{math}.
	Así, el mínimo es
	\begin{math}
		\left\|A\overline{x}-b\right\|=
		\left\|
		\begin{bNiceMatrix}
			3 & -2 \\
			0 & 3  \\
			4 & 4
		\end{bNiceMatrix}
		\begin{bNiceMatrix}
			0.76 \\
			0.6
		\end{bNiceMatrix}-
		\begin{bNiceMatrix}
			3 \\
			5 \\
			4
		\end{bNiceMatrix}
		\right\|
		\approx 6.9457.
	\end{math}
\end{frame}

\begin{frame}[fragile]
	\begin{minipage}{0.45\textwidth}
		\inputminted[
			fontsize=\scriptsize,
		]{text}{resultado_pregunta1.txt}
	\end{minipage}
	\begin{minipage}{0.45\textwidth}
		\begin{listing}[H]
			\inputminted[
				fontsize=\scriptsize,
				breaklines,
			]{python}{pregunta1.py}
		\end{listing}
	\end{minipage}
\end{frame}
% \begin{frame}
% 	\begin{enumerate}\setcounter{enumi}{1}
% 		\item

% 		      Use transformaciones de Householder para triangulizar la matriz:
% 		      \begin{equation*}
% 			      A=
% 			      \begin{bNiceMatrix}
% 				      2 & 2 & 4  & 18 \\
% 				      1 & 3 & -2 & 1  \\
% 				      3 & 1 & 3  & 14
% 			      \end{bNiceMatrix}.
% 		      \end{equation*}
% 	\end{enumerate}

% 	\begin{solution}
% 		Sea $A_{1}=A$.
% 		Entonces,
% 		\begin{equation*}
% 			v_{1}
% 			=
% 			a_{1}+
% 			\operatorname{signo}\left(a_{11}\right)
% 			\left\|a_{1}\right\|
% 			e_{1}              \\
% 			v_{1}
% 			=
% 			\begin{bNiceMatrix}
% 				2 \\
% 				1 \\
% 				3
% 			\end{bNiceMatrix}+
% 			\operatorname{signo}\left(2\right)
% 			\left\|
% 			\begin{bNiceMatrix}
% 				2 \\
% 				1 \\
% 				3
% 			\end{bNiceMatrix}
% 			\right\|
% 			\begin{bNiceMatrix}
% 				1 \\
% 				0 \\
% 				0
% 			\end{bNiceMatrix}
% 			=
% 			\begin{bNiceMatrix}
% 				2 \\
% 				1 \\
% 				3
% 			\end{bNiceMatrix}+
% 			\sqrt{2^{2}+1^{2}+3^{2}}
% 			\begin{bNiceMatrix}
% 				1 \\
% 				0 \\
% 				0
% 			\end{bNiceMatrix}=
% 			\begin{bNiceMatrix}
% 				5.74165739 \\
% 				1          \\
% 				3
% 			\end{bNiceMatrix}.
% 		\end{equation*}
% 		Recordemos,
% 		\begin{definition}[Transformación de Householder]
% 			Sea $w\in\mathbb{R}^{n}$ con $w^{t}w=1$.
% 			Entonces la matriz $n\times n$
% 			\begin{equation*}
% 				P = I -2 ww^{t}
% 			\end{equation*}
% 			recibe el nombre de \alert{transformación de Householder}.
% 		\end{definition}
% 	\end{solution}
% \end{frame}
\end{document}