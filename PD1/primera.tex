%! Aldo Luna, Alexandra Mazzetti, Carlos Aznarán, Edward Canales
%! Universidad Nacional de Ingeniería
%! Facultad de Ciencias
%! Lima, Perú
%! Uso:
%! $ sudo pacman -Syu texlive-most zathura # instalar las dependencias y un visor
%! $ arara primera
%! $ zathura primera.pdf
%! Ver https://wiki.archlinux.org/title/TeX_Live
% arara: clean: {
% arara: --> extensions:
% arara: --> ['aux','log','nav',
% arara: --> 'out','snm','toc','pdf']
% arara: --> }
% arara: lualatex: {
% arara: --> interaction: batchmode
% arara: --> }
% arara: clean: {
% arara: --> extensions:
% arara: --> ['aux','log','nav',
% arara: --> 'out','snm','toc']
% arara: --> }
\PassOptionsToPackage{svgnames}{xcolor}
\documentclass[
	spanish,
	8pt,
	utf8,
	xcolor=table,
	handout,
	aspectratio=169,
	professionalfonts,
	notheorems,
	mathserif,
	% t
]{beamer}
\setbeamersize{text margin left=0pt,text margin right=0pt}
\usepackage[spanish,es-sloppy]{babel}
\spanishdatedel\decimalpoint
\usepackage{mathtools}
\usepackage{enumerate}
\usepackage{multicol}
\usepackage{array}
\usepackage[linesnumbered,ruled,boxed,vlined,spanish]{algorithm2e}
\usepackage{algorithmicx}

\newcolumntype{x}[1]{>{\centering\arraybackslash\hspace{0pt}}p{#1}}

\newcounter{savedenum}
\newcommand*{\saveenum}{\setcounter{savedenum}{\theenumi}}
\newcommand*{\resume}{\setcounter{enumi}{\thesavedenum}}

\setbeamertemplate{navigation symbols}{}
\setbeamertemplate{footline}{}
\setbeamertemplate{headline}{}

\begin{document}

\begin{frame}
	\begin{enumerate}
		\item
		      Sea $\left\{a_{n}\right\}_{n\in\mathbb{N}}$ la sucesión
		      recurrente definida por

		      \begin{equation*}
			      a_{1}=1,\quad
			      a_{n+1}=\sqrt[3]{4+{\left(a_{n}\right)}^{2}}.
		      \end{equation*}

		      Demuestre que

		      \begin{multicols}{2}
			      \begin{enumerate}[a)]

				      \item

				            $\forall n\in\mathbb{N}:a_{n}\leq2$.

				      \item

				            $\left\{a_{n}\right\}_{n\in\mathbb{N}}$ es
				            convergente y determine su límite.
			      \end{enumerate}
		      \end{multicols}

		\item
		      Sea $\left\{a_{n}\right\}_{n\in\mathbb{N}}$ la sucesión
		      definida por

		      \begin{equation*}
			      a_{1}=\sqrt{2},\quad
			      a_{n}=\sqrt{2a_{n-1}}.
		      \end{equation*}

		      Demuestre que $\left\{a_{n}\right\}_{n\in\mathbb{N}}$ es
		      convergente y determine su límite.

		\item
		      Sea $\left\{a_{n}\right\}_{n\in\mathbb{N}}$ la sucesión
		      definida por

		      \begin{equation*}
			      a_{1}=7,\quad
			      a_{n+1}=\sqrt{\frac{{\left(a_{n}\right)}^{2}+2}{a_{n}+2}}.
		      \end{equation*}

		      Demuestre que

		      \begin{multicols}{2}

			      \begin{enumerate}[a)]
				      \item

				            $\forall n\in\mathbb{N}:a_{n}\geq1$.

				      \item

				            $\left\{a_{n}\right\}_{n\in\mathbb{N}}$ es
				            convergente y determine su límite.
			      \end{enumerate}
		      \end{multicols}

		      \saveenum
	\end{enumerate}
\end{frame}

\begin{frame}
	\begin{enumerate}
		\resume

		\item
		      Sean $\left\{a_{n}\right\}_{n\in\mathbb{N}}$ la sucesión
		      definida por

		      \begin{equation*}
			      a_{1}=1,\quad
			      a_{n}=4n+a_{n-1},
		      \end{equation*}
		      y $b_{n}\coloneqq\dfrac{a_{n}}{2n^{2}}$.
		      Demuestre que

		      \begin{multicols}{2}

			      \begin{enumerate}[a)]
				      \item

				            $\forall n\in\mathbb{N}:\left|a_{n}-2n^{2}\right|<2n$.

				      \item

				            Estude la acotación de
				            $\left\{b_{n}\right\}_{n\in\mathbb{N}}$ y
				            determine su límite.
			      \end{enumerate}
		      \end{multicols}

		\item
		      Sea $\left\{a_{n}\right\}_{n\in\mathbb{N}}$ la sucesión
		      definida por

		      \begin{equation*}
			      a_{1}=\sqrt{3},\quad
			      a_{n}=\sqrt{3a_{n-1}}.
		      \end{equation*}

		      Determine $\lim\limits_{n\to\infty}a_{n}$.

		\item
		      Sea la sucesión $\left\{x_{n}\right\}_{n\in\mathbb{N}}$
		      definida por

		      \begin{equation*}
			      x_{n}=\cos\left(\frac{1}{n}\right),
		      \end{equation*}

		      converge linealmente a $1$, y
		      $\left\{p_{n}\right\}_{n\in\mathbb{N}}$ con

		      \begin{equation*}
			      p_{n}=
			      x_{n}-
			      \frac{
			      {\left(x_{n+1}-x_{n}\right)}^{2}}{
			      x_{n+2}-2x_{n+1}+x_{n}
			      }=
			      \cos\left(\frac{1}{n}\right)-
			      \frac{
				      {\left(
						      \cos\left(\frac{1}{n+1}\right)-
						      \cos\left(\frac{1}{n}\right)
						      \right)}^{2}
			      }{
				      \cos\left(\frac{1}{n+2}\right)-
				      2\cos\left(\frac{1}{n+1}\right)+
				      \cos\left(\frac{1}{n}\right)
			      },
		      \end{equation*}

		      que es el método $\Delta^{2}$ de Aitken,
		      compruebe que converge linealmente a $1$ más rápidamente.

		      % \begin{algorithm}[H]
		      %   $A\leftarrow 1.0$\;
		      %   $B\leftarrow 1.0$\;
		      %   $p\leftarrow 0$\;
		      %   \Mientras{\normalfont $\left(\left(A+1\right)-A\right)-1=0$}{
		      %     $A\leftarrow 2\ast A$\;
		      %     $p\leftarrow p+1$\;
		      %     \Mientras{\normalfont $\left(\left(A+B\right)-A\right)-B\neq0$}{
		      %       $B\leftarrow B+1$\;
		      %     }
		      %   }
		      % \end{algorithm}

		      \saveenum
	\end{enumerate}
\end{frame}

\begin{frame}
	\begin{enumerate}
		\resume

		\item

		      Sea la sucesión $\left\{x_{n}\right\}_{n\in\mathbb{N}}$
		      definida por

		      \begin{equation*}
			      x_{n}=2^{-n}.
		      \end{equation*}

		      Determine los valores de la sucesión usando el método
		      $\Delta^{2}$ de Aitken.

		\item

		      Sea la sucesión
		      $\left\{x_{n}\right\}_{n\in\mathbb{N}\cup\{0\}}$ definida
		      por

		      \begin{equation*}
			      x_{0}=\frac{1}{10},\quad
			      x_{n}=\frac{1}{4}e^{n}.
		      \end{equation*}

		      Determine los valores de la sucesión usando el método
		      $\Delta^{2}$ de Aitken.

		\item

		      Sea la sucesión
		      $\left\{x_{n}\right\}_{n\in\mathbb{N}\cup\{0\}}$ definida
		      por

		      \begin{equation*}
			      x_{0}=\frac{1}{2},\quad
			      x_{n}=e^{-n}.
		      \end{equation*}

		      Determine los valores de la sucesión usando el método
		      $\Delta^{2}$ de Aitken.

		\item

		      Sea la sucesión
		      $\left\{u_{n}\right\}_{n\in\mathbb{N}\cup\{0\}}$ definida
		      por

		      \begin{equation*}
			      u_{0}=\frac{3}{2},\quad
			      u_{1}=\frac{5}{3},\quad
			      u_{n+1}=2003-\frac{6002}{u_{n}}+\frac{4000}{u_{n}u_{n-1}}.
		      \end{equation*}

		      Determine los valores de la sucesión usando el método
		      $\Delta^2$ de Aitken.

		      \saveenum
	\end{enumerate}
\end{frame}

\begin{frame}
	\begin{enumerate}
		\resume

		\item

		      Implemente un programa en \texttt{python} para convertir
		      los siguientes números binarios con signo de $8$ bits a
		      base $10$.

		      \begin{multicols}{4}

			      \begin{enumerate}[a)]
				      \item

				            $\left(10111000\right)_{2}$.

				      \item

				            $\left(01010101\right)_{2}$.

				      \item

				            $\left(11111111\right)_{2}$.

				      \item

				            $\left(00011011\right)_{2}$.
			      \end{enumerate}
		      \end{multicols}

		\item

		      Implemente un programa en \texttt{python} para convertir
		      los siguientes números decimales a binarios con signo de
		      $8$ bits.

		      \begin{multicols}{4}
			      \begin{enumerate}[a)]
				      \item

				            $56$.

				      \item

				            $85$.

				      \item

				            $127$.

				      \item

				            $27$.
			      \end{enumerate}
		      \end{multicols}

		\item

		      Implemente un programa en \texttt{python} un registro de
		      $5$ bits.
		      Encuentre la suma de los siguientes números usando el
		      complemento a $2$.


		      \begin{multicols}{3}
			      \begin{enumerate}[a)]
				      \item

				            $-1011$ y $-0101$.

				            \

				      \item

				            $+0111$ y $-0011$.



				      \item

				            $+0011$ y $-0101$.

				            \

				      \item

				            $+0100$ y $-0111$.



				      \item

				            $-0011$ y $-0101$.

				            \
				      \item

				            $-0111$ y $-0010$.
			      \end{enumerate}
		      \end{multicols}

		      \saveenum
	\end{enumerate}
\end{frame}

\begin{frame}
	\begin{enumerate}
		\resume
		\item

		      Dado el siguiente sistema numérico de punto flotante, con
		      las siguientes características

		      \begin{multicols}{3}

			      \begin{enumerate}[i)]
				      \item

				            Una base $\beta$,

				      \item

				            $n$ dígitos de la mantisa,

				      \item

				            un exponente $m\leq e\leq M$,
			      \end{enumerate}
		      \end{multicols}

		      donde $\left\{\beta,n,m,e,M\right\}\subset\mathbb{Z}$.
		      Defina la forma normalizada de un número y en especial la
		      del cero.
		      Halle el número total de números que se puede expresar en
		      este sistema.

		\item

		      Del sistema de punto flotante, halle el número cuyo valor
		      absoluto es el más pequeño, su sucesor inmediato y la
		      distancia entre ellos.


		\item

		      Demuestre que $\frac{4}{5}$ no se puede representar de
		      manera exacta como número de máquina.
		      ¿Cuál será el número de máquina más cercano?

		\item

		      Encuentre el número de máquina de 32 bits que está a la
		      derecha de $\frac{1}{9}$.

		\item

		      Determine el mayor, menor elemento positivo, números de
		      elementos del conjunto $\mathbb{F}\left(10,6,-9,9\right)$,
		      así como las siguientes operaciones $x+y$, $x-y$, $xy$ y
		      $\frac{x}{y}$, donde $x=\pi=3.141592653589\ldots$ e
		      $y=e=2.7182818284590\ldots$.

		\item Justificando su respuesta, determine el valor de verdad de
		      las siguientes proposiciones.

		      \begin{enumerate}[a)]
			      \item

			            Si la cantidad de elementos del conjunto
			            $\mathbb{F}$, con $\mathbb{F}(2, t .-1,2)$ es $33$,
			            entonces los dígitos en la mantisa son $4$.

			      \item

			            En un computadora de doble precisión su sistema de
			            números puntos flotantes está distribuido en $11$
			            bits para la mantisa y $52$ para el exponente.

		      \end{enumerate}


		\item

		      Probar que el cardinal del sistema de números de punto
		      flotante normalizado $\mathbb{F}\left(\beta, t, L, U\right)$ es


		      \begin{equation*}
			      \operatorname{card}\left(\mathbb{F}\right)=
			      2\left(\beta-1\right)\beta^{t-1}\left(U-L+1\right)+1.
		      \end{equation*}

		      En particular, para $\mathbb{F}\left(10,3,-2,3\right)$,
		      calcular $x_{\min}$, $x_{\max}$, $\epsilon_{M}$.

	\end{enumerate}
\end{frame}

\end{document}