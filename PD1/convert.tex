%! Carlos Aznarán, Cristhian Caleb
%! Universidad Nacional de Ingeniería
%! Facultad de Ciencias
%! Lima, Perú
%! Uso:
%! $ sudo pacman -Syu texlive-most zathura # dependencias, visor
%! $ arara primeras
%! $ zathura primera.pdf
%! Ver https://wiki.archlinux.org/title/TeX_Live
% arara: clean: {
% arara: --> extensions:
% arara: --> ['aux','log','nav',
% arara: --> 'out','snm','toc','pytxcode','pdf']
% arara: --> }
% arara: lualatex: {
% arara: --> shell: yes,
% arara: --> draft: yes,
% arara: --> interaction: batchmode
% arara: --> }
% arara: biber
%! arara: pythontex
% arara: lualatex: {
% arara: --> shell: yes,
% arara: --> draft: yes,
% arara: --> interaction: batchmode
% arara: --> }
% arara: lualatex: {
% arara: --> shell: yes,
% arara: --> synctex: yes,
% arara: --> interaction: batchmode
% arara: --> }
% arara: clean: {
% arara: --> extensions:
% arara: --> ['aux','log','nav',
% arara: --> 'out','snm','toc','pytxcode']
% arara: --> }
\PassOptionsToPackage{svgnames}{xcolor}
\documentclass[
	spanish,
	8pt,
	utf8,
	xcolor=table,
	handout,
	aspectratio=169,
	professionalfonts,
	% notheorems,
	mathserif,
	leqno,
	% t
]{beamer}
\setbeamersize{text margin left=5pt,text margin right=5pt}
\usepackage[spanish,es-sloppy]{babel}
\spanishdatedel\decimalpoint
\usepackage{mathtools}
\usepackage{minted}
\usepackage{enumerate}
\usepackage{multicol}
% \usepackage{pythontex}
\usepackage[
	citestyle=numeric,
	style=apa,
	backend=biber,
	defernumbers=true,
	sorting=ynt,
	maxcitenames=4
]{biblatex}
\addbibresource{references.bib}

\newcolumntype{x}[1]{>{\centering\arraybackslash\hspace{0pt}}p{#1}}

\newcounter{savedenum}
\newcommand*{\saveenum}{\setcounter{savedenum}{\theenumi}}
\newcommand*{\resume}{\setcounter{enumi}{\thesavedenum}}

\setbeamertemplate{navigation symbols}{}
\setbeamertemplate{footline}{}
\setbeamertemplate{headline}{}

% https://tex.stackexchange.com/questions/68080/beamer-bibliography-icon
\setbeamertemplate{bibliography item}{%
	\ifboolexpr{ test {\ifentrytype{book}} or test {\ifentrytype{mvbook}}
		or test {\ifentrytype{collection}} or test {\ifentrytype{mvcollection}}
		or test {\ifentrytype{reference}} or test {\ifentrytype{mvreference}} }
	{\setbeamertemplate{bibliography item}[book]}
	{\ifentrytype{online}
		{\setbeamertemplate{bibliography item}[online]}
		{\setbeamertemplate{bibliography item}[article]}}%
	\usebeamertemplate{bibliography item}}

\defbibenvironment{bibliography}
{\list{}
	{\settowidth{\labelwidth}{\usebeamertemplate{bibliography item}}%
		\setlength{\leftmargin}{\labelwidth}%
		\setlength{\labelsep}{\biblabelsep}%
		\addtolength{\leftmargin}{\labelsep}%
		\setlength{\itemsep}{\bibitemsep}%
		\setlength{\parsep}{\bibparsep}}}
{\endlist}
{\item}

\title{
	\huge\sffamily
	Primera Práctica Calificada\quad Grupo N$^{\circ}$~8
}

\subtitle{
	\large\scshape
	Análisis y Modelamiento Numérico I\quad CM4F1 B\\
		\normalsize\normalfont
		Profesor Jonathan Munguia La Cotera
}

\author{
	Carlos Alonso Aznarán Laos\quad\and\quad
  Cristhian Caleb Blas Huaroc
}

\institute{\large
	Facultad de Ciencias \and
	Universidad Nacional de Ingeniería
}
\date{5 de octubre del 2022}

\begin{document}

\frame{\titlepage}

\section{1. Calculadora decimal binario}

\begin{frame}
	\frametitle{\secname}
	\begin{enumerate}
		\item

		      Calculadora que convierta un número en sistema decimal a sistema binario.

		\item

		      Calculadora que convierta un número en sistema binario a sistema decimal.
	\end{enumerate}

	\begin{solution}
		Debemos asegurar que $x_{n}$ es convergente y al menos del tipo lineal.
		\begin{description}
			\item[$x_{n}$ converge a $0$.]

				Vemos que:
		\end{description}
	\end{solution}
\end{frame}

\section{2. Calculadora decimal complemento a 1}

\begin{frame}
	\frametitle{\secname}
\end{frame}

\section{3. Calculadora decimal complemento a 2}

\begin{frame}
	\frametitle{\secname}
\end{frame}

\section{4. Calculadora decimal octal binario}

\begin{frame}
	\frametitle{\secname}
\end{frame}

\section{5. Calculadora decimal hexadecimal binario}

\begin{frame}
	\frametitle{\secname}
\end{frame}

\section{6. Calculadora decimal binario fraccionario}

\begin{frame}
	\frametitle{\secname}
\end{frame}

\section{7. Calculadora decimal a flotante preción simple IEEE-754}

\begin{frame}
	\frametitle{\secname}
\end{frame}

\section{8. Calculadora decimal a flotante precisión doble IEEE-754}

\begin{frame}
	\frametitle{\secname}
\end{frame}

\begin{frame}

	\begin{solution}
		\begin{minipage}{0.55\textwidth}
			% \url{https://docs.python.org/3/library/functions.html\#int}.
			\begin{listing}[H]
				\inputminted[
					fontsize=\footnotesize,
					breaklines,
				]{python}{pruebas.py}
			\end{listing}
		\end{minipage}
		\begin{minipage}{0.35\textwidth}
			hola hola hola
		\end{minipage}
	\end{solution}
\end{frame}

\begin{frame}
	\begin{alertblock}{Cambio de base}
		Sean $r$ y $s$ dos bases cualesquiera.
		Hay dos métodos que son generales.
		\begin{equation*}
			37.75_{\left(10\right)} = 100101.11_{\left(2\right)}
		\end{equation*}
		\begin{equation*}
			r\longrightarrow s
		\end{equation*}
		Vamos a operar en base decimal.
		La que opera en la base de origen se conoce como división y multiplicación por la base.
		Es útil cuando queramos pasar de base 10 a cualquier otra base, por ejemplo, base 2.
		Utiliza divisiones para la parte entera y multiplicaciones para la parte fraccionaria.

		El método de sustitución en serie opera en la base de destino.
		Es útil cuando queramos pasar de cualquier otra base, por ejemplo, base 2 hacia la base 10.


	\end{alertblock}
\end{frame}

\begin{frame}\transblindsvertical
	\frametitle{Referencias}
	%------------------------------------------------------------ 1
	\only<1>{
		\begin{itemize}
			\item Libros
			      \nocite{*}
			      \printbibliography[heading=none,keyword=book]
		\end{itemize}
	}
	%------------------------------------------------------------ 2
	\only<2>{
		\begin{itemize}
			\item Artículos científicos
			      \printbibliography[heading=none,keyword=paper]
		\end{itemize}
	}
	%------------------------------------------------------------ 3
	\only<3>{
		\begin{itemize}
			\item Sitios web
			      \printbibliography[heading=none,keyword=online]
		\end{itemize}
	}
\end{frame}

\end{document}